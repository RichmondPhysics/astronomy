\chapter{Are We Alone?}


As far as we can tell, our Sun is a fairly typical star in a fairly
typical galaxy.  The Universe contains a huge number of stars very much
like our own.  Do any of them have life?  Nobody knows, but that doesn't
stop us from trying to find out!

First, let's get a feel for the number of stars in the Universe.
Here are some useful numbers to get us started:

\begin{itemize}
\item There are about 300 billion stars in our own Milky Way Galaxy.
\item The observable Universe is a sphere with a radius of about
5000 megaparsecs.
\item There's about one large galaxy like our own for every 100 cubic
megaparsecs (Mpc$^3$) of volume.
\end{itemize}

About how many large galaxies are there in the observable Universe?

\vskip 1.5in

Assuming all of those large galaxies have about the same number
of stars as our own Galaxy, about how many stars are there in the
observable Universe?

\vskip 1in

Suppose that a star has a one-in-a-million chance of having
a planet that develops life.
How many stars in our Galaxy have life?

%\vskip 1in
\newpage

Continuing to suppose that a star has a one-in-a-million chance
of developing life, how many stars in the observable Universe have life?

\vskip 1in

Now suppose that life is less likely: say that only one in a trillion
($10^{12}$) stars develops life.  How many stars in the observable Universe
have life?

\vskip 1in

Of course, the problem is that we have no idea what the probability
of life developing around any given star is.  It could be
one in a million, one in a trillion, or something much larger or much smaller!

Suppose that there are planets out there with life on them.  How could we
detect them?  If we were really lucky, maybe the creatures on them
would have developed a technological society similar to ours.  
In that case, they might actually send out radio broadcasts
into the Galaxy, to see if anyone's out there.  Even if they
didn't do that, 
they might use radio waves to communicate with each other (just 
as we do).  In that case, we
might be able to detect their broadcast signals.  (Just as an alien
civilization 20 light-years away might just now be receiving our radio
and television broadcasts from 20 years ago.  I wonder if they like
``Seinfeld.'')

Back in 1961, the astronomer Frank Drake wrote down an equation that
can be used to estimate the number of alien civilizations that we might
hope to detect in this manner.  Here's one way of writing the {\it
Drake Equation}:
$$
N_c=N_*f_pn_ef_lf_if_cf_L.
$$
Here's what this means.

The quantity $N_c$ is the number of civilizations in our Galaxy
from which we might possibly be able to detect signals.  That's what
we want to find.  The things on the right are
\begin{itemize}
\item $N_*$ is the number of stars in our Galaxy.
\item $f_p$ is the fraction of all of those stars that have planets.
\item $n_e$ is the average number of planets around each such star
that can sustain life.  (One way people often interpret this term
is that it's the number of planets with liquid water.  It's much
easier for the complex chemical reactions that cause life to occur
if there's liquid water around.)
\item $f_l$ is the fraction of all of such planets on which life actually
evolves.
\item $f_i$ is the fraction of all such life-containing planets on which
{\it intelligent} life arises.
\item $f_c$ is the fraction of all such intelligent-life-containing
planets where the inhabitants are communicating with us.  (This could mean
either the fraction that are deliberately beaming signals out into space
to let us know they're there, or the fraction that are using 
broadcast signals to communicate with each other.)
\item $f_L$ is the fraction of the Galaxy's life during which
the civilization is communicating.  The life of the Galaxy so far is
about 10 billion years, so this last term is the lifetime of the
technologcal civilization divided by 10 billion years.
\end{itemize}

Although this equation looks complicated and 
has a lot of terms in it, I hope it makes sense where all the terms come from.
Most of the terms (all the ones called $f$) are fractions that could
be anywhere from 0 to 1.

The problem is that we have very little idea what a lot of the numbers
on the right side of this equation should be!  

For instance, maybe
life evolves pretty much every time it possibly can, so that every
planet that can possibly sustain life does in fact evolve life.
If that's true, then $f_l$ would be about equal to 1 (that is, 100\%).  
On the other hand,
maybe it takes an incredible stroke of luck for the molecules on a planet
to organize themselves in such a way that life gets started.  If that's
true, then $f_l$ would be an extremely small number.

Similarly, it might be the case that intelligent life is very rare
-- maybe there are lots of planets oozing with bacteria but very few
with big-brained creatures like us.  In that case, $f_i$ would be small.
Or maybe once life gets started it's inevitable that it grows in complexity
until something intelligent arises.  In that case, $f_i$ would be nearly equal
to 1.

Using Google (or some other search engine), dig around the web for a while
to come up with a range of plausible estimates for these numbers.\footnote{Just
in case there's any doubt, let me mention that ``dig around the web''
is not generally considered to be part of a good scientific methodology.
In this case, though, we just don't know what the right answers are,
and I just want to get an idea of a plausible range of answers.}
  (``Drake
equation'' is a good search term.  Be warned: there are slightly
different ways of writing the equation, with different choices
of symbols for the various quantities.)  Feel free to use your own
judgment along with anything you find on the web.
Let's assume
that the number of stars in our Galaxy is known to be 300 billion:
$$
N_*=3\times 10^{11}.
$$
For each of the other numbers, list two values: a ``pessimistic'' estimate
and an ``optimistic'' one.  There's not one right answer for these,
of course!

\begin{itemize}
\item $f_p=$
\item $n_e=$
\item $f_l=$
\item $f_i=$
\item $f_c=$ 
\item $f_L=$ 
\end{itemize}

Using all of your low numbers, estimate the number $N_c$ of communicating
civilizations.  Then use all of your high numbers to estimate the number.
Give a range of plausible values for $N_c$:

\vskip 1in

There is an ongoing project called the Search for Extraterrestrial
Intelligence (SETI) looking for radio signals from intelligent beings
on other planets.  Based on your estimates, do you think SETI is likely
to succeed?  

