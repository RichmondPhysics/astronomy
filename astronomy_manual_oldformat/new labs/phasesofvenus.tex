\documentclass[twoside]{report}
\renewcommand{\chaptername}{Lab}
\usepackage{graphicx}
\usepackage{epsf}
\textwidth 6in
\textheight 9in
\topmargin -0.2in
\oddsidemargin 0.3in
\evensidemargin -0.3in
\parindent 0pt
\parskip 6pt





\begin{document}
\chapter*{Lab 6$1\over 2$ \\ Phases of Venus}


One of the most important things Galileo observed with his telescope is that
Venus has phases like the Moon.   This was the ``smoking gun'' that showed
that the earth-centered (pre-Copernicus) system couldn't possibly be right,
so it went a long way toward convincing people that the Earth really did
go around the Sun.  We're going to examine the phases of Venus to see what
Galileo saw and why it mattered.

Start up {\it Starry Night}, and configure it as follows:  
\begin{itemize}
\item Set the time to noon today.  Set the time step to ``days.''
\item Turn off 
daylight, so that it's possible to see the stars and planets when the
Sun is up.  (Remember that you can do this by clicking on the ``View Options''
tab on the left and unchecking the ``Daylight'' option under ``Local View.'').
\item Turn on labels for the planets by checking the ``Labels'' box next to
``Planets-Moons'' in the ``View Options'' menu.  
\item Set the view to be centered on the Sun (right-click on the Sun and 
choose ``centre'').  
\end{itemize}

Let time run forwards and backwards for a year or two and observe the
motion of Venus.  Note that it swings back and forth past the Sun,
never getting more than a certain angular separation from the Sun in the sky.
Once you've done this, reset the date to today.

Here's one preliminary question before we look at the phases of Venus.  If
you wanted to observe Venus today, at what time of night should you observe?
Specifically, find a rough time when Venus is above the horizon but the Sun
is below the horizon.  The easiest way to do this is to set the time step
to hours and step forward or backward one hour at a time.

Best time to observe Venus these days:

\vskip 0.7in

Note that you've got a fairly narrow window of time when the Sun is
down but Venus is up, so it'd be pretty hard to observe Venus right
now.  The easiest time to observe Venus is when its as far away from
the Sun in the sky as possible.  This is called the time of ``maximum
elongation.''  Set the time back to noon, set the time step to days,
and run time backwards until you find the most recent time of maximum
elongation.  (Hint: it's some time in March 2006.)  Measure the angular
separation between
Venus and the Sun at that time.  (Remember that you can do that by 
dragging the mouse from one object to the other, but be sure the pointer
looks like an arrow, not a hand, when you start.)

Maximum angular separation between Sun and Venus:

\vskip 0.7in

Here's a diagram showing the Earth's and Venus's orbits around the
Sun.  Suppose that the Earth is at the uppermost point in its orbit as
shown, and suppose that Venus is at maximum elongation from the Sun
(so that it appears as far from the Sun as possible in the sky).  Draw
a circle to mark the position of Venus in its orbit.  (There are
actually two possibilities.  One corresponds to the case where Venus
is to the East of the Sun and one to the case where it's to the West.
If you assume everything orbits
counterclockwise, you can figure out which which, although
it's a bit tricky.  For the moment, it doesn't matter which one you choose.)

\begin{figure}[h]
\centerline{\epsfxsize 3in\epsfbox{figs/venus1.eps}}
\end{figure}


Venus, like the Moon, shines by reflected sunlight.  That means that
the only part of Venus we'll see is the part that's illuminated by the 
Sun.  In the diagram above, shade in the half of Venus that's lit up by
the Sun, and then use the diagram to predict the phase of Venus
at this time.  (That is, will Venus appear like a crescent, like
a ``full'' Venus, or what?)

\vskip 0.7in

After you've made your prediction, use {\it Starry Night} to test it.
Center the view on Venus and zoom in to enlarge the image of Venus.
Does it have the phase you predicted?

\vskip 0.7in

Now set the date back to today, and 
zoom in on Venus to observe its phase.  What is the phase of Venus today?

\vskip 0.7in

Based on your observation, is Venus in front of the Sun or behind the Sun
today?

\vskip 0.7in

Zoom back out again, and center the field of view on the Sun.  Let time
run forward until Venus has gone through about half of one orbit.  You
should see it pass by the Sun, go out to maximum elongation, and then come
back until it approaches near the Sun again.  At this point, when
Venus's path is about to cross past the Sun again, is Venus in front
of the Sun or behind the Sun?  (Don't use {\it Starry Night} to answer this; 
use what you know about the orbits.)

\vskip 0.7in

Based on your answer to the previous question, what do you expect the 
phase of Venus to be?

\vskip 0.7in

What do you expect about the angular size of Venus: should it be bigger
or smaller than the last time you looked at it?

\vskip 0.7in

Center the view on Venus and zoom in to test these two predictions.
Did they come out right?

\vskip 0.7in

Keep the field of view centerd on Venus and zoomed all the way in.  Let time
run forward fast for a year or two.  The main things to note are that
Venus goes through a full set of phases (from crescent to full and back),
and that its angular size changes along with its phase.

Earlier, I said that this observation provided strong proof that
the old Earth-centered theory was wrong.  Let's see why.  Remember that
in the old theory the Earth was at the center, the Sun went around the
Earth, and Venus moved on an epicycle between the Earth and the Sun like this:

\begin{figure}[h]
\centerline{\epsfxsize 3in\epsfbox{figs/venus2.eps}}
\end{figure}

At each of the points A,B,C,D in the diagram, what would the phase
of Venus be, according to this model?

A:

B:

C: 

D:

What phases of Venus would {\it never} occur in the earth-centered
model?

\vskip 0.7in

The fact that Venus is observed to have those phases is the ``smoking
gun'' I referred to.

One more thing.  Now that you've measured the maximum elongation of
Venus, you can use it to figure out the radius of Venus's orbit.
Sketch a picture showing Earth, Sun, and Venus at the moment of 
maximum elongation (this will look the same as the diagram on the
second page).  

\vskip 2in

Connect these three bodies with straight lines to form a triangle.
When Venus is at maximum elongation, this triangle will have a right
angle at Venus.  The angle at the Earth is the angular separation
between Venus and Sun (that is, the angle you determined on the second page).
Indicate both of those angles on the diagram.

We also know the length of one side of the triange: the distance from
the Earth to the Sun is 1 AU.  Mark this on your triangle as well.

Now you have a right triangle, and you know one angle (other
than the right angle) and one side.
That means you can use trigonometry to find the lengths of the
other sides.  Determine the radius of Venus's orbit trigonometrically.
(If you don't remember how to do this, ask me.)


\end{document}