\chapter{The Hertzsprung-Russell Diagram}


In this lab, you'll use the data from the Hipparcos satellite
to construct a Hertzsprung-Russell diagram.  The data you'll
use is the real stuff returned by the satellite; all I've done
for you is to convert it from archaic astronomy units (such as
magnitudes) into standard modern physics units (such as watts per
square meter).

The full Hipparcos data set contains 100,000 stars, but you'll be 
working with only the approximately 
1000 closest stars and the approximately 1000 brightest stars.

\begin{enumerate}

\item There are two Excel files on Blackboard (in the ``Downloads for Labs'' 
section) containing Hipparcos data
on the closest and brightest stars.  Download these files to the desktop,
and open the file containing the brightest stars.  It contains
about 1000 rows and four columns: Catalog number, Parallax angle, 
temperature, and brightness.  Note that the units of the various
quantities are indicated on the second line (except for the catalog
number, which doesn't have any units). 
You will be adding a bunch of new columns to the spreadsheet.  As you
do, use those first two lines indicate what quantity is in each column
and what its units are.

\item For each of these stars, you will need to determine its luminosity
and radius.  Since there are 1000 stars, you wouldn't want to do it by
hand, so you'll use Excel formulae instead.  However, it's good to 
calculate at least one by hand in order to check the Excel calculation
you're going to do.  So for the first star in the list, calculate
the following things:
\begin{enumerate}
\item Distance in parsecs.
\item Distance in meters.
\item Luminosity.
\item Radius.
\end{enumerate}

\vskip 3in

\item Now use Excel formulae to repeat these calculations.
Column E will contain the distances to the stars in parsecs.  Indicate
this in the first two rows of the column.  Then, in cell E3, enter
a formula that calculates the distance to the first star.  
(Appendix C contains advice on how to do this.)  If your
formula agrees with the result you calculated above, then drag
the formula down to fill in the distances to the rest of the stars.

\item In the next three columns, use formulae to determine distance
in meters, luminosity, and radius.  

\item Create a Hertzsprung-Russell diagram for this set of stars.
This is just a graph with temperature on the horizontal axis and
luminosity on the vertical axis.  (See Appendix C if you need
advice on making graphs in Excel.)  

When you first make your graph
it will probably not look right.  One reason for this is that H-R diagrams
traditionally have {\it logarithmic} $y$ axes.  This means that
the numbers on the $y$ axis are spaced out in powers of 10.  
Double-click on the $y$ axis of your graph to get the ``Format Axis''
window, and check the box that says ``Logarithmis scale.''
In
that same window, you can
adjust the minimum value on the axis so that the points aren't all
crowded together near the top.

Also, by annoying, stupid tradition, H-R diagrams are plotted with
high temperatures on the left instead of the right.  So double-click
on the $x$ axis and check the ``values in reverse order'' box.

By the time you're done, your graph should look something like a
normal H-R diagram.  In particular, the main sequence should be clearly
visible running from upper left to lower right.  Also, the $x$ and $y$ 
axes should be labeled to indicate both what quantity
is being plotted ({\it e.g.}, ``Temperature'') and what units it's
in ({\it e.g.} ``K'').

\item Print out your graph.  Mark one more point on it by hand
to indicate the Sun.  The Sun's temperature is 5800 K and its
luminosity is $3.86\times 10^{26}$ W.  The Sun should lie
on the main sequence; if it doesn't, something has gone wrong.

\item Follow the same steps to create an H-R diagram for the data
set consisting of the closest stars.  Print out this graph as well.

\item Which of the two data sets contains more white dwarfs?
Explain why this makes sense.

\vskip 1in

\item Which of the two data sets contains more giants?  Explain why
this makes sense.

\vskip 1in

\item For the
closest stars, make a graph showing the temperatures of the stars on the $x$
axis and the radii on the $y$ axis.  Use a logarithmic scale on the
$y$ axis, and make sure the ranges on the axes are set so that the points
aren't all crowded together at one end.  As in your previous graphs,
be sure the axes are clearly labeled including units.
Print out the graph.

\item What is the radius of a main-sequence star with a temperature of 5000 K?
What is the radius of a white dwarf star with a temperature of 10,000 K?
What is the radius of the largest star in this sample?

\vskip 1in


\end{enumerate}




