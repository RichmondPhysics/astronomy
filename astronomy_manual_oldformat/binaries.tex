\chapter{Binary Stars}


\paragraph{Introduction}.

The most important thing about binary star systems is that they provide
a way to determine the masses of stars.  In this lab you will use
simulated observations of a binary star system to calculate stellar
masses.

The star system in this lab is a visual binary system, which means that
the two stars can be resolved separately in a telescope.  From Earth,
our view of the orbits of the stars is ``edge-on'' -- that is, we're seeing
the orbit from the side, rather than looking down on it.  In a system
like this, the orbits of the two stars are sometimes moving towards
us and sometimes away from us.  This is good, because it allows us
to use the Doppler effect to measure the stars' speeds.

(Incidentally, when a binary star system is viewed from a {\it perfectly}
edge-on perspective, the two stars pass directly in front of each other,
creating an eclipsing binary system.  In today's lab, we'll assume that
the alignment of the system is not quite perfect enough to cause eclipses
like this.)


\paragraph{Procedure.}

Start up the {\it Virtual Astronomy Laboratory} software, and select
{\it Visual Binary Stars} (number 22).  You should see an animation
of two stars orbiting each other.  Use the {\it Viewing Angle}
menu to switch back and forth between two viewing perspectives
on this system.  We'll assume that our view from Earth is the ``Planar''
or ``Side View'' perspective.  Select that view.  Also, go to
the {\it Grid} menu and turn on the grid.

To analyze this system, we'll need to know how far away it is from us.
We'll assume that that has already been measured.  For some reason,
this information is contained only in the {\it Check your answers}
window, so click on {\it Check your answers} (even though we don't have any
answers yet) to find out the distance to the star.  Another useful
bit of information in this window is the angular scale of the grid.  For
future reference, record both the distance to the star and the
number of arc-seconds corresponding to each square on the grid here:

\vskip 1in

Close the {\it Check your answers} window for now.

The next thing we'll need to know is the radius (semimajor axis)
of the stars' orbits.  To find this out, stop the animation when the stars are
at their maximum separation from each other.  Use the grid to 
estimate as accurately as you can the stars' angular separation
from each other.  Then use the small-angle formula to determine
the actual separation between the stars.  This final result is
the radius (semimajor axis) of the orbit (the thing we usually call $a$).

\vskip 1.5in

The natural next step would be to measure the period of the stars'
orbit (that is, the time it takes them to go once around).  Then we could
use Kepler's third law to determine the total mass.  We're going
to use a different line of attack, though: we're going to determine
the stars' speeds instead, and use that to figure out the masses.

Make sure that the animation is stopped at a time when the two stars
are at their maximum separation from each other.  Click on
one of the two stars.  You should see a graphic indicating the spectrum
of the star.  The colored bars represent the four main hydrogen lines
in the spectrum.  (This spectrum is misleading in appearance: 
it makes it look like the star has an emission spectrum, when in
reality these would be absorption lines.)  

We want to measure the
Doppler shift in the spectral lines.  The shift is very small, so 
to see it we have to zoom in on the a part of the spectrum right near
one of the lines.  Click on any one of the four spectral lines to do this.
You should see two copies of this spectral line.  
The one on the top
is a spectrum measured in the lab (with no Doppler shift), and the
other is the spectrum of the star.  
(If you don't
see both of these, it's possible that the star's spectral line
is shifted all the way off the scale in this view.  If that
happens, choose another spectral line.)

Use the grid to measure the
difference in wavelength $\Delta\lambda$ between these two lines,
and then use the Doppler effect formula
$$
{\Delta\lambda\over\lambda_0}={v\over c}
$$
to determine the star's speed.

\vskip 2in

Repeat the procedure with one of the other spectral lines.  You should
get a different $\Delta\lambda$, but you should end up with
(very nearly) the same speed $v$.

\vskip 2in

Repeat the procedure for the other star: use two of its spectral lines
to determine its speed twice, and make sure the two speeds you get
are consistent with each other.

\vskip 2in

Before proceeding further, you might want to check your results
so far.  In the {\it Check your answers} window, 
you can enter your value of $a$ and the speeds of the two stars.
Leave the masses blank for now.  If the computer doesn't
confirm that you've got the correct answers, talk to me.

Now let's get back to finding the stars' masses.
Call the speeds of the two stars $v_1$ and $v_2$.  Add them together
to get the speed of one star relative to the other:
$v_{\rm rel}=v_1+v_2$.   (If you called one of the two $v$'s negative
because of the direction the star was moving, ignore that minus sign
and just add the magnitudes together.)

\vskip 1in


The speed $v_{\rm rel}$ is what you would measure if you stood
on the surface of one star and measured how fast the other star was
going relative to you.  That speed is related to the radius $a$ 
and the period $P$ of the
stars' orbits like this:
$$
v_{\rm rel}={2\pi a\over P}.
$$
The reason is that, from star 1's point of view, star 2 moves around
in a circle of radius $a$.  The distance the star travels during one
complete orbit is the circumference $2\pi a$.  Dividing distance by time
gives speed.

Using
the speed $v_{\rm rel}$ and the radius $a$, which you've already determined,
find the period $P$ of the stars' orbit:

%\vskip 2in
\vfil\eject


Now you can use Kepler's third law, in the form
$$
P^2={4\pi^2 a^3\over G(m_1+m_2)}
$$
to determine the total mass $(m_1+m_2)$ of the two stars:

\vskip 2in

The last thing we want to do is to decide how this total mass is distributed
between the two stars.  To do that, we go back to the speeds.
The more massive star is the one that moves less in its orbit,
and the less massive star moves more.  In fact, the speeds of the
two stars scale inversely with their masses (if one is twice
as heavy, it moves half as fast). Mathematically, we can express
this by saying that the product of mass and speed is the same
for the two stars:
$$
m_1v_1=m_2v_2.
$$
Use this fact, along with your determination of the total mass $m_1+m_2$,
to figure out the masses $m_1$ and $m_2$.  

\vskip 2in

Use \textit{Check your answers} to see if you got the masses right. If
not, talk to me.

