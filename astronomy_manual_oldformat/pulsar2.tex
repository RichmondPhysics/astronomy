\chapter{The Crab Pulsar}


In this lab you'll use the same software as last time to examine the 
rotation period of the Crab pulsar.  In particular, you'll measure the
rate at which the pulsar is slowing down.
\bigskip\bigskip

{\bf Part 1: Period of the Crab Pulsar.}

Start up the radio telescope, turn on tracking, and use the Hot List to
point the telescope at the Crab pulsar (known to its friends as ``0531+21'').
Start up a receiver, and observe the pulses from this star.  Because this star
pulses very quickly, you'll need to set the {\bf Horz.\ Secs} level to 2.0.
Adjust the gain to some appropriate value.  Turn on the {\bf Record} button,
and start the receiver.  Let it run for at least four full traces across
the screen (8 seconds).  Once you have acquired your data, open it up
in the data analysis window.  You will use this data to determine the
period of the pulsar as accurately as possible.

First, zoom in as far as you can on the leftmost portion of the data.
Determine the time of the very first pulse (by dragging the vertical
cursor over the peak of the pulse).  The software is supposed to start
the clock right at the moment of the first pulse, so the time of this pulse
should be zero.  If this is not the case, talk to me.

Let's call that very first pulse ``pulse number zero.''  We'll
number all the pulses after it 1,2,3, etc.
Scroll forward to pulse number 20 and measure its time:

\medskip
$$
\mbox{Time of Pulse \# 20}=\hbox to 2in{}
$$
\medskip

Use this value to calculate the period of the pulsar:

\medskip
$$
\mbox{Period of Pulsar}=\hbox to 2in{}
$$
\medskip

Our goal today is to measure the {\it change} in the period
of the pulsar over time.  This change is very tiny, so we're going to
need to be highly accurate.  The number you just calculated for the period
isn't quite good enough.  

We could improve our measurement of the period if we count more pulses.
For instance, if you had measured the time of pulse \# 200 instead
of pulse \# 20, you'd get a better result.  It's tedious to count
200 pulses, though, so we'll use a shortcut.

Based on the value you found for the period, estimate the time at 
which pulse number 50 should occur.  (Don't count 50 pulses -- just
predict when it should occur.)

\medskip
$$
\mbox{Predicted time of Pulse \# 50}=\hbox to 2in{}
$$
\medskip

Now scroll ahead through your data to test this prediction.  Position
the cursor at the time you just calculated, and see if there's a pulse 
there.  You should find that there's a pulse at almost, but not exactly,
the time you predicted.  Measure the actual time of that pulse
as accurately as possible:

\medskip
$$
\mbox{Actual time of pulse \# 50}=\hbox to 2in{}
$$
\medskip

Note: you don't have to count the pulses to make sure this really is
\# 50.  Since you made a pretty accurate prediction of when \# 50
should occur, and you found a pulse at just about that time, you can
be confident you've got the right pulse even without counting.

Using the value you just measured the time of pulse \# 50, determine
the period of the pulsar.  This should be nearly the same as the
value you got before, but the new value is more accurate:

\medskip
$$
\mbox{Period of Pulsar (2nd estimate)}=\hbox to 2in{}
$$
\medskip

Record this value to at least four significant figures (for instance,
0.03283 seconds, although this isn't the right value).

We just ``leapfrogged'' from a 20-pulse estimate of the period to a 
50-pulse estimate.  Let's play that game one more time to get a 200-pulse
estimate.  That'll be accurate enough for our purposes.

Use your most recent estimate of the period of the pulsar to predict
when pulse \# 200 should occur:

\medskip
$$
\mbox{Predicted time of Pulse \# 200}=\hbox to 2in{}
$$
\medskip

Then go to that portion of your data, and find the pulse that's closest
to that time.  This is the

\medskip
$$
\mbox{Actual time of Pulse \# 200}=\hbox to 2in{}
$$
\medskip

From this value, determine yet another estimate of the

\medskip
$$
\mbox{Period of Pulsar (3rd estimate)}=\hbox to 2in{}
$$
\medskip

Record this value to at least five significant figures.

In principle, we could continue this ``leapfrog'' procedure, but
there's no need: this value is an accurate enough estimate.  There's
just one more bit of information we'll need: the uncertainty in this
measurement of the period (that is, an estimate of how far off
this number might be).  

One way to determine this is to measure the period several different times
and see how much it changes from measurement to measurement.  You
just determined the period by using the time of pulse \# 200.  
Determine it again using pulse \# 199 and pulse \# 201:

\medskip
\begin{eqnarray*}
\mbox{Time of Pulse \# 199}&=&\hbox to 2in{}\\
\mbox{Period determined from Pulse \# 199}&=&\hbox to 2in{}\\
\mbox{Time of Pulse \# 201}&=&\hbox to 2in{}\\
\mbox{Period determined from Pulse \# 201}&=&\hbox to 2in{}\\
\end{eqnarray*}
\medskip

Now that you have three different determinations of the period (from pulses
199,200,201), average them together to get your best estimate of the
true period.  Then figure out the uncertainty associated with this
value.  The uncertainty is how far off the worst of your three
measurements was from the mean.

Give your final answer here, in the form ``({\it Number}
 $\pm$ {\it Number}) seconds,''  with the two numbers being your
best estimate and its uncertainty.  Your best estimate should have five
significant figures.

\medskip
$$
\mbox{Final Value for Period (with uncertainty)}=\hbox to 2in{}
$$

\bigskip\bigskip

{\bf Part 2. The Period of the Pulsar in the Future.}

To see whether the Crab pulsar is slowing down, we need to wait
a while and measure the period again.  Since the slowdown is very
gradual, we need to wait a long time.  Five years works well.
Rather than actually waiting five years, which would be inconvenient,
we can tell the software to pretend that five years have gone by
and simulate data from a future date.

Close down the data analysis window and the receiver window.  Under
the {\bf File} menu, click on {\bf Date/Time}, and change the date to
five years from today.  After you've done that, you'll need to tell
the telescope to point toward the Crab pulsar again.

Now {\it repeat the entire procedure} to determine an accurate measurement
of the period of the pulsar, together with its uncertainty.

\newpage

\ \ \ 
\vfill

Based on the values you determined, {\it and their uncertainties},
can you conclude that the Crab
pulsar is slowing down?  (Is the difference between the two periods so great
that it lies outside the bounds of the measurement uncertainties?)

\vskip 1in

What is the rate at which the period is changing, in seconds per year?

\vskip 1in

The Crab pulsar was created in a supernova explosion in the year 1054.
Assuming that it has been slowing down at a steady rate ever since that
time, what was its period just after it was created?

\vskip 1in
\eject
