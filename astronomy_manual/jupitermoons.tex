\chapter{Kepler's Third Law}


Kepler's third law says that the period of a planet's orbit
(the time to orbit once, called $P$) and the radius
of its orbit (called $a$) are related like this:
$$
P^2=ka^3.
$$
Here $k$ is a constant.  Its numerical value depends on what units
we choose for $P$ and $a$, but it's the same for all of the planets.

Jupiter and its moons look like a mini-solar system: the moons orbit
Jupiter in pretty much the same way that the planets orbit the Sun.
So it's natural to wonder whether Jupiter's moons obey Kepler's third
law.  In this lab, you'll use 
{\it Stellarium} to simulate
observations of the moons to test Kepler's third law.

Start up \textit{Stellarium}. Turn off the effects of Earth's
atmosphere / daylight, and make the Earth transparent, so that we can see
Jupiter all the time. (Once again, we're making our lives a bit easier than
it is for real astronomers: in reality, we could only make
the sort observations in this lab at certain times of year.)
Set the program to the equatorial (stationary-sky) mount.
Find Jupiter, center it in the field
of view, and set the field of view to a size where you can clearly 
see Jupiter and its four Galilean moons (Io, Europa, Callisto, Ganymede).

Let time run forward at a pretty high speed and observe the orbits of 
the moons.  The orbits are really very close to circular, but we're
viewing them from the side, so the moons appear to move back and forth.

Pick one of the moons, and run time forwards and backwards until the
moon is at its greatest separation from Jupiter.  (Be as accurate as
possible; you can probably determine the time to within about 10 minutes
for the fastest-moving moon and less than an hour for the slowest one.)
Measure the angular separation between Jupiter and the moon at this moment.
Repeat the procedure for all four moons.  Record your results in the table
on the last page.  You'll initially get the angular separation
in arc-minutes and arc-seconds.  Convert your answer to a number of 
arc-seconds.  For instance, if you got a value of $2'$ and $20''$, you
would record $140''$ (because $1'$ is $60''$, and $2\times 60+20=140$)..

How far is Jupiter from us at the time these observations are being taken?
Record the distance in A.U. here:

\vskip 1in

Use the small-angle formula to convert the angular separations
for all four moons into actual separations.  These will be the
maximum distances the four moons get from Jupiter; in other words,
they'll be the radii of the moons' orbits.  Record them in the data table,
along with their units.

Now that you know the radii of the orbits, the next step is to find the
periods.  The most accurate way to do this is to measure the time between
moments when the moon passes in front of or behind Jupiter.

Set the date back to today, and then advance time
forwards or backwards until one of the moons is just
passing in front of the edge of Jupiter from the left.  (Aim for
an accuracy of about 5 minutes.)  Record
the the date and time, and also the ``Modified Julian Date (MJD).''
See Appendix \ref{app:stell} for details on how to do this.

Let time advance until that moon
is just passing in front of the left edge of Jupiter again (one complete
orbit later).  Record the date, time, and MJD again.  

Next, you should determine the time that elapsed
between these two measurements.  This is the period of the moon's orbit.
Give the result as a decimal number of days. (You can do this by working
out the difference between the two dates and times, but it's much easier
using the MJDs. In fact, that's what MJDs are for!)

Repeat this procedure for all four moons. Each time you start working on a
new Moon, set the time back to the present day.

Now that you know the period and the radius of all four moons,
you can test Kepler's third law.  Square all of the periods to get $P^2$,
and cube all the radii to get $a^3$.  Take the ratio $P^2/a^3$.
Kepler's third law says that this should be the same for all four
moons.  Is it?

\vskip 1in

A couple of followup questions:

1. Is the numerical value of $k$ for Jupiter's moons the same
as the value for the planets orbiting the Sun?  To answer this,
figure out the value of $k$ for the Earth's orbit around the
Sun using the same units as you used for Jupiter's moons.

\vskip 3in

2. Why did we have to use the moons of Jupiter for this lab,
instead of using Earth's Moon?

\newpage


\begin{center}
\begin{tabular}{|l|l|l|l|l|}
\noalign{\hrule}
{\bf MOON} & Io & Callisto & Ganymede & Europa \\
\noalign{\hrule}
Max. angular separation & \qquad\qquad\qquad\qquad & \qquad\qquad\qquad\qquad &
\qquad\qquad\qquad\qquad &\qquad\qquad\qquad\qquad \huge\strut\\
from Jupiter ($'$ and $''$) & & & &\huge\strut \\
\noalign{\hrule}
Max. angular separation & & & & \huge\strut\\
from Jupiter ($''$) & & & & \huge\strut\\
\noalign{\hrule}
Radius of orbit & & & & \huge\strut\\
& & & & \huge\strut\\
\noalign{\hrule}
Date/time of first & & & & \huge\strut\\
crossing of Jupiter &  & & & \huge\strut\\
\noalign{\hrule}
MJD of first crossing  & & & & \huge\strut\\
& & & & \huge\strut\\
\noalign{\hrule}
Time of second & & & & \huge\strut\\
crossing of Jupiter &  & & & \huge\strut\\
\noalign{\hrule}
MJD of second crossing  & & & & \huge\strut\\
& & & & \huge\strut\\
\noalign{\hrule}
Orbital period & & & & \huge\strut\\
& & & & \huge\strut\\
\noalign{\hrule}
$P^2$ & & & & \huge\strut\\
& & & & \huge\strut\\
\noalign{\hrule}
$a^3$ & & & & \huge\strut\\
& & & & \huge\strut\\
\noalign{\hrule}
$k$ & & & & \huge\strut\\
& & & & \huge\strut\\
\noalign{\hrule}
\end{tabular}
\end{center}

