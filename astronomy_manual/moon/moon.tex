\section{The Moon}

\makelabheader

\bigskip

In this lab, you'll check a bunch 
of things about the motion and phases of the Moon.
%Although most of the things in this lab can be done in \textit{Sky
%Safari}, there are one or two things that can't, so we'll 
%use the more powerful \textit{Stellarium} on the lab PCs.



%% \medskip

%% {\bf A. Getting familiar with Stellarium}

%% Since we haven't ued \textit{Stellarium} yet in class, 
%% spend a few minutes playing around with it to learn how it works.
%% Pretty much everything you've done so far in \textit{Sky Safari} can
%% also be done in \textit{Stellarium}. In particular, try out the following.
%% Consult Appendix B for details about how to do these things.

%% \begin{enumerate}
%% \item Adjust the time and date. When you first start the 
%% program, it will show the sky at the current time.  It's much
%% more interesting to see the sky at night. You can
%% also examine what things look like at different times of year.

%% \item Speed up the flow of time by a large amount, so that
%% a day goes by in just a few seconds.

%% \item Initially, the program is showing you the view of the sky
%% from here in Richmond. Switch to some other locations.

%% \item Click on a star or other astronomical object.
%% Information about that body should appear on the screen.
%% Some of that information may not make sense yet, but it will!

%% \item Drag the image around  to look at different parts 
%% of the sky.

%% \item Zoom in to look at a small patch of the sky, then
%% zoom back out
%% (using the mouse wheel or the page-up/page-down keys).
%% Note the display at the bottom that says ``FOV.'' This indicates
%% the field of view, which as you know 
%% indicates the size of the patch of sky
%% that's visible on the screen at that moment. 

%% \item Click on the ``find'' icon and search for various celestial
%% objects. 
%% Try Mars, for example, or Polaris.


%% \item The pop-up menus in the lower left
%% contain a bunch of buttons you can click to change
%% the appearance in various ways. I've listed the ones I think
%% are most useful in the Appendix. Try these out. Some are harder
%% to understand than others.

%% \end{enumerate}


\bigskip

{\bf A. Sidereal and Synodic Months}

A {\it sidereal month} is the length of time it takes the
Moon to make one complete motion around the sky with respect to the
stars.  In other words, at the end of a sidereal month, the Moon
appears in the sky next to the same stars as it did at the beginning
of the month.
A {\it synodic month} is the time it takes the Moon to go through one
full cycle of phases (from full Moon to full Moon, for example).
You'll use {\it Stellarium} to measure the length of both sidereal and
synodic months.

Start up {\it Stellarium}.  Make the ground transparent and turn
off the Earth's atmosphere, so that we can see what the Moon
is doing at all times. This makes our lives much easier
than the real-life observers who first figured all this stuff out:
they could only see the Moon when it was above the horizon at night.
Again, consult the Appendix to see how to do these things, and of course
ask me if you can't figure it out.

First, step forward in time until the Sun and Moon are right
next to each other in the sky. This is the next time when
the phase of the Moon will be new.
Determine, to within an accuracy of about an hour or so, the
moment when the Moon and Sun are closest to each other in
the sky.
Record the date and time:

\answerspace{1in}

Now let time run forward for a bit less than a month, until the phase of the Moon is nearly new
again.  Just as before, find the moment, accurate to the nearest hour,
when the Moon and Sun are closest together.  Record the date and time:

\answerspace{1in}

Based on these results, what is the length of a synodic month?
Give your answer as a decimal number of days, with at least
one digit after the decimal point.

\answerspace{1in}

\pagebreak[1]
Now you'll figure out the length of a sidereal month.  Click on a star
that's very near the Moon, and keep the field of view centered
on that star. Record the date and time
when that star passes closest to the Moon, accurate to the
nearest hour.  

\answerspace{1in}

Then let the time advance
for a bit less than a month, until you see the marked star pass close to the Moon
again.  Record the time it passes closest to the Moon, accurate
to the nearest hour.

\answerspace{1in}

Find the length of a sidereal month as a decimal number of days.

\answerspace{1in}

The year is 365.24 days long.  How many synodic months are there 
in a year? (Give your answer as a decimal number with at least one
digit after the decimal place).

\answerspace{1in}

How many sidereal months are there in a year?

\answerspace{1in}

You should find that the difference between these two numbers is
very close to 1.  (If you don't find this, let me know.)  This
is a general rule relating sidereal and synodic periods for
all satellites.  You might enjoy trying to figure out why
it's true.  (Then again, you might not.)

\bigskip

\pagebreak[1]

{\bf B. Phases of the Earth}

\medskip

{\bf Question:} 
Suppose the Moon is in a waxing crescent phase as seen from the
Earth.  An astronaut on the Moon looks at the Earth.  What
phase of the Earth does she see?

I want you to figure out the answer to this question on your
own first, and then test it with {\it Stellarium}.

Draw a diagram showing Earth, Sun, and Moon when the Moon is
a waxing crescent.  Indicate the directions of the Earth's orbit
around the Sun and the Moon's orbit around the Earth.

\answerspace{2in}

Based on this diagram, what is the answer to the question above?
(Your answer should be something like ``full Earth,'' ``waning
crescent Earth,'' ``waxing gibbous Earth.'')

\answerspace{1in}

Now test your answer.  Adjust the time until the phase of the Moon
is waxing crescent.  Then click on the Location icon, and
change the location to the Moon.
Adjust the view until you see the 
Earth.  Does its appearance agree with your prediction?  (For
instance, if you predicted the Earth would be in a crescent phase, is it?)
Allow time to run forward.  Is the Earth waxing or waning?
Does this agree with your prediction?

