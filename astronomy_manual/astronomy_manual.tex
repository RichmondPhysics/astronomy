\documentclass[english,twoside]{article}

\input{../../131/StudentGuideModule1/labmanual_formatting_commands} %all general latex packages, commands, and definitions now here.
\newcommand{\coursefolder}{Phys121} %This defines the place students will look for various files
\newcommand{\coursesection}{\\Spring 2022\\Section 01, 9:00 am, with Ted Bunn} %This defines the section (and/or year) for the title page.
\ForceSectionOddPage %This option makes each lab start on odd numbered page (right hand side).

%Use the line below to disable hyperlinks, to make sure no markups are visible around references when printing.
%\hypersetup{draft}

%Use the following line to print instructor notes at the end of the document
%\includeinstructornotestrue

%The \includeonly line below is a great way to save time so you don't always have to compile the WHOLE latex document, if for instance you've only made changes to a single lab.  If you want to compile more than two labs, the syntax is \includeonly{lab1,lab2,lab3} with no spaces after the commas.
%The master.pdf produced will have only the title page, TOC, and that single lab, though the other lab names will appear in the TOC.
%\includeonly{angularsize/angularsize}

%Use the following line to override all of the 1's and 0's in the \includelab statements below
%\includealllabstrue

\makeindex

\begin{comment}
Notes to Ted:
1. You should rewrite the two paragraphs on the title page to your liking
2. Do you have a cool picture to use for the cover?
3. Should we include the (new!) Vernier Scale Appendix?
\end{comment}

\begin{document}



\title{Physics For Doing!\\
Activities for Astronomy}

%\author{Matthew G. Belk}
\author{Ted Bunn}
%\author{Mirela Fetea\footnote{Current address: Germanna Community College, Fredericksburg, VA}}
%\author{Gerard P. Gilfoyle}
%\author{Henry Nebel}
%\author{Philip D. Rubin\footnote{Current address: Department of Physics, George Mason University, Fairfax, VA}}
%\author{Shaun Serej}
%\author{Jack Singal}
%\author{Matthew L. Trawick}
%\author{Michael F. Vineyard\footnote{Current address: Department of Physics, Union College, Schenectady, NY}}
\affil{Department of Physics, University of Richmond, VA}

\maketitle

\vspace{0.8 in}

%\begin{abstract}

\begin{center}
\large{\textbf{Welcome to Astronomy!}}
\end{center}

The exercises in this manual have been developed to support an investigative
physics course that emphasizes active learning. 
%The units are made up of activities designed to guide your investigations in the laboratory. 
Your written work will consist primarily of documenting
your class activities by filling in the entries in the spaces provided
in the units. The entries consist of observations, derivations, calculations,
and answers to questions. Although you may use the same data and graphs
as your partner(s) and discuss concepts with your classmates, all
entries should reflect your own understanding of the concepts and
the meaning of the data and graphs you are presenting. Thus, each
entry should be written in your own words. It is very important
to your success in this course that your entries reflect a sound understanding
of the phenomena you are observing and analyzing. 

Some of these exercises
have been taken from the Workshop Physics project at Dickinson College
and the Tools for Scientific Thinking project at Tufts University
and modified for use at the University of Richmond. Others have been
developed locally. 
We wish to acknowledge the support we have received for this project
from the University of Richmond and the Instrumentation and Laboratory
Improvement program of the National Science Foundation. 
%\end{abstract}


\newpage
\
\thispagestyle{plain}

\newpage
\
%\setcounter{page}{2} %changed from 1 to 2 on 6/9/15 by MT.  The reason is that in the circuit labs, the odd pages need to be on the right (the fronts of each sheet of paper) and the even pages need to be on the left.  This is critical, because these labs have some pages that are meant to be cut out with scissors, so the backs have to be left blank.  This is done by using the \cleardoublepage command, which requires that the odd/even pages not be reversed from the usual.




\tableofcontents
\cleardoublepage

%--------------------------------------------
\part{Seeing and Measuring Things}

\includelab{1}{localdistance/localdistance}
\includelab{1}{angularsize/angularsize}
\includelab{1}{resolution/resolution}

%--------------------------------------------
\part{The Nighttime Sky and the Solar System}

\includelab{1}{stellarium1/stellarium1}
\includelab{1}{solarcell/solarcell}% !TEX pdfSinglePage
\includelab{1}{navigation_stell/navigation_stell}
\includelab{1}{moon/moon}
\includelab{1}{retrograde/retrograde}
\includelab{1}{periods/periods}
\includelab{1}{jupitermoons/jupitermoons}
\includelab{1}{phasesofvenus/phasesofvenus}
\includelab{1}{kepler/kepler}

%--------------------------------------------
\part{Light and Optics}
	
\includelab{1}{light/light}
\includelab{1}{diffraction/diffraction}
\includelab{1}{hydrogen/hydrogen}
\includelab{1}{lenses1/lenses1}
\includelab{1}{solar/solar}

%--------------------------------------------
\startappendix

\includelab{0}{appendices/skysafari/skysafari}
\includelab{1}{appendices/stellarium/stellarium}
\includelab{0}[../../131/StudentGuideModule1/]{appendices/treatment_data}
\includelab{0}[../../131/StudentGuideModule1/]{appendices/datastudio/datastudio}
\includelab{0}[../../131/StudentGuideModule1/]{appendices/capstone/capstone}
%\includelab{1}[../../131/StudentGuideModule1/]{appendices/excel/excel}
\includelab{0}[../../131/StudentGuideModule1/]{appendices/video_analysis_tracker/video_analysis_tracker}
\includelab{0}[../../131/StudentGuideModule1/]{appendices/one_page_uncertainty/one_page_uncertainty}
\includelab{0}[../../131/StudentGuideModule1/]{appendices/instrumentation/instrumentation}
\includelab{0}[../../132/StudentGuideModule2/]{appendices/nuke_safety/nuke_safety}

% The following command prints the "Instructor Notes" section at the end of the manual.
\ifincludeinstructornotes \startinstructornotes \fi
\end{document}
