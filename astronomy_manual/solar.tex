\chapter{Solar Radiation and the Greenhouse Effect}

\bigskip\bigskip

The Earth (like the other planets) is warmed by
sunlight.  The Earth also radiates energy into
space, mostly in the form of infrared radiation.
On average, the amount of radiation received from the Sun
must equal the amount of radiation emitted by the planet.  After 
all, if the Earth emitted less radiation into space than it 
received from the Sun, then its temperature would keep going up;
and if it emitted more radiation than it received, it would keep
cooling down.  Since the temperature of the Earth is fairly stable
over the years, the radiation coming in from the Sun must balance
the radiation being emitted.

Note: You've doubtless heard about global warming, so you may
wonder how I can get away with saying that the temperature
of the Earth is fairly stable. Global warming does mean there's
an imbalance between the incoming and outgoing radiation, but
it's very slight. Although that slight difference is enough to have
major consequences for the climate, it's small enough that we can safely
ignore it in the calculations we're about to do.

In this lab, you will use this idea to form an estimate of the
temperature of the Earth.  Specifically, you will calculate
the total amount of solar power striking the Earth's surface.
Then you will assume that the Earth radiates that same amount
of power back into space.  Assuming that the Earth radiates
as a blackbody, you can use the Stefan-Boltzmann law to 
relate the radiated power to the temperature and so calculate
the temperature of the Earth.

This calculation depends on one important assumption: that the
Earth radiates like a blackbody.  That assumption is wrong,
so our calculated temperature will not match the actual
temperature of the Earth.  In particular, a phenomenon
known as the ``greenhouse effect'' blocks some of the
Earth's radiation from escaping into space.  The results
of this calculation will give an indication of how important
the greenhouse effect is here on Earth.

Let's get started.
Remember that the Sun's luminosity is
$$
L_{\odot}=3.86\times 10^{26}\,{\rm W}.
$$
A watt (W) is a unit of power equal to a joule per second.
This is the total amount of power emitted by the Sun in all directions.
The first thing we want to do is to figure out how much of this
power strikes the Earth.

Consider an imaginary sphere centered on the Sun with a radius
of 1 A.U.  All of the radiation that leaves the Sun must pass through
this sphere.  Calculate the amount of radiation that strikes each square
meter of this imaginary sphere's surface by dividing the Sun's luminosity
by the sphere's surface area.  Remember that the surface area of a sphere
of radius $R$ is
$$
A=4\pi R^2.
$$
Remember that an A.U. is $1.49\times 10^{11}$ meters.
Give your answer in units of watts per square meter (W/m$^2$).

\vskip 1in

Now work out how much solar radiation strikes the Earth.  To do this,
note that the Earth covers up a small circular patch of that great big
imaginary sphere.  The amount of radiation striking the Earth will equal
the amount striking that small circular patch.  Recall that the area
of a circle is $\pi R^2$.  The radius of the Earth is 6380 kilometers.
What is total amount of solar radiation
striking the Earth?  Give your answer in watts.

\vskip 1.5in

The Earth doesn't actually absorb all of that energy; some of it
is reflected into space.  The fraction of energy that is reflected
by an object is called the object's {\it albedo}.  Measurements of Earth
from space have shown that the Earth's albedo is 0.35, meaning that it
reflects 35\% of all the radiation that hits it back into space and absorbs
the rest.  

What is the total amount of solar radiation {\it absorbed} by the Earth?

\vskip 1in

This number must equal the amount of radiation emitted by the Earth.
Assume for the moment that the Earth emits radiation as a blackbody.
That means that it obeys the Stefan-Boltzmann law,
$$
F=\sigma T^4.
$$
Recall that $\sigma=5.67\times 10^{-8}$ W/(m$^2$ K$^4$).
Also, remember that the flux $F$ in the Stefan-Boltzmann law
is the power radiated per area (watts per square meter).  To get
the total power radiated by the Earth, therefore, we should
multiply the flux by the surface area of the Earth.  As noted
earlier, the surface
area of a sphere is $4\pi R^2$, so the total
power radiated by the Earth
is
$$
L_{\oplus}=4\pi R^2\sigma T^4.
$$
Here $\oplus$ is the symbol for the Earth.  $L$ stands
for ``luminosity,'' which is what we always call the total
amount of power emitted by an object.  $R$ is the radius
of the Earth, and $T$ is the temperature.

Set this expression equal to the total amount of solar power
being absorbed by the Earth and solve for the temperature $T$.
Your answer will come out in kelvin.  Subtract 273 to convert 
it to degrees Celsius.

\vskip 1.5in

Does this seem like a reasonable estimate of the Earth's 
actual average temperature?

\vskip 1in

The main thing that went wrong in this calculation is the assumption
that the Earth radiates as a blackbody, also known
as a  ``perfect radiator.''
Suppose that something in the Earth's atmosphere is blocking
radiation from getting out, so that it's not a perfect radiator.
Then the amount of radiation leaving the Earth will be less than 
the amount calculated from the Stefan-Boltzmann law.  This
alters the balance between incoming and outgoing radiation, resulting
in an increase in temperature.

Carbon dioxide and other gases in the Earth's upper atmosphere reflect some of
the Earth's infrared radiation back down to the planet, preventing it from
escaping into space.  This ``greenhouse effect'' means that
the Earth is not a perfect radiator, and it is why
the Earth is warmer than your calculation above would indicate.

The greenhouse effect is responsible for climate change, so
we think of it as a bad thing these days,
but without it, the Earth would be uninhabitably cold!  The
problem we face now is that human activity has led to
an {\it increase} in the
greenhouse effect, altering the balance that existed for 
many millennia before we started altering the composition of the
atmosphere.

Venus is an interesting example of the greenhouse effect.
Follow the same steps as above to figure out what
the surface temperature of Venus would be in the absence of
a greenhouse effect.  Here's some useful information about Venus:

\begin{itemize}
\item Distance from Sun to Venus: 0.723 A.U.
\item Radius of Venus: 6050 km.
\item Venus's albedo: 0.65.
\end{itemize}

You'll also need to know the luminosity of the Sun ($3.86\times 10^{26}$ W)
and the number of meters in an A.U. ($1.49\times 10^{11}$ m).

\vfil

Probes of Venus indicate that its surface temperature is about 700 K
(430$^\circ$C, or 800$^\circ$F).  Would you say that the
greenhouse effect is more important on Venus than on Earth or 
vice versa?

\vskip 1in
\eject




