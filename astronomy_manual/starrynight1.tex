\chapter{Daily Motion of Stars}

This lab has a few purposes:
\begin{itemize}
\item To get used to some features of the {\it Starry Night}
program.
\item To identify some prominent constellations and asterisms
that can be seen from our location.
\item To see how stars move in the night sky.
\end{itemize}

\bigskip

{\bf A. Messing around.}

Start up {\it Starry Night}, and spend a few minutes playing around with
some of its features.  Here are some of the more useful things
to mess around with.  

\begin{enumerate}
\item Adjust the time and date in the upper left corner.  You can highlight
any of the numbers, as well as the AM/PM display, and adjust them with
the arrow keys or by typing in new numbers.  When you first start the 
program, it will show the sky at the current time.  It's much
more interesting to see the sky at night.
\item The box that initially says ``1x'' controls how fast time flows
in the program.  Switch it to something much faster, like ``3000x''
to see a sped-up movie of the sky.  Also, try changing that
setting to ``hours.''  This causes the view of the sky to
move forward by one hour in each frame of the movie.  The playback
buttons to the right of this box let you play the movie forwards or
backwards, stop it, or advance it one frame at a time.
\item To the right of that is a panel that lets you change the
observer's location.  It should initially say ``Richmond.''  Click on that,
go to ``Other,'' and choose a new location for the observer.
Once you're done messing around with that, go back to Richmond.
\item Point the cursor right at a star or other astronomical body.
Information about that body should appear on the screen.
Some of that information may not make sense yet, but it will!
\item If you double-click on any object, or right-click and select 
``Show Info,'' you'll get even more information.
\item When the cursor looks like a hand, you can change the direction
that you're looking by dragging the image around.  Try looking east
or north instead of south, for instance.
\item The slider on the upper right adjusts the width of your field
of view.  Initially, it's set to view a large chunk of the sky.
Slide it to the right and you'll focus in on smaller and smaller
patches of sky.
\item The left side of the screen has a bunch of pop-up menus.
The most useful one is ``View Options.''  This turns on and off
a bunch of features in the display.  In particular, try
checking various boxes such as ``Stick Figures,'' ``Labels,''
and ``Boundaries'' in the ``Constellations'' section.
\item The other useful pop-up display is ``Find.''  If you go there, and type
in the name of an astronomical object, it will show you where
that object is in the sky.  Try Mars, for example, or Polaris (which
is the name of the North Star).
\end{enumerate}

\bigskip

{\bf B. Big Dipper, Mizar, and Alcor.}

Once you're done messing around, set your location back to Richmond,
looking north, with the largest possible field of view.  Set the
time to 10:00 tonight.

You should see the Big Dipper in the lower left of your field of view.  Turn
on the constellation labels and stick figures.  Notice that the Big
Dipper is not a full constellation; it's just part of the larger constellation
Ursa Major (the Great Bear).  A group of stars like the Big
Dipper, which is easily
identifiable and has a name, but which isn't a whole constellation,
is called an ``asterism.''  

The second star in the handle of the Big Dipper is called Mizar.
Right-click on this star and choose ``Centre'' to move it to the center
of the field of view.  Then zoom in on this star until you can see
another star called Alcor right near it.  Find the angular separation
between these two stars by clicking on one of them (make sure that the
cursor looks like an arrow, not a hand, when you click) and dragging
to the other one.  What is the angular separation between
these two stars?

\vskip 1in

This separation is visible to the naked eye, if you have good eyesight.
Next time you're out on a clear night, look to see if you can spot both
Alcor and Mizar.

Display information about Mizar.  Note that it's listed as a 
multiple star system.  Magnify the view until you can see
Mizar's companion.  (It'll be called ``Zeta Ursae Majoris.'')
Find the angular separation between the two
stars by clicking and dragging from one star to the other.

\vskip 1in

Show info on Mizar to determine how far away it is from Earth.
Use the small-angle formula to determine the separation between
Mizar and its companion star.  The small-angle formula will give you
an answer in light-years.  Convert this to meters and to astronomical
units.

\vskip 2in

Amazingly enough, each of these two stars is itself a binary star system!
Unfortunately, these two pairs are too close together for us to 
see them separately, even with our best telescopes.  (How do we know
that they're there, then?  Good question!)  So when you look at Alcor and
Mizar in the night sky, you're really looking at at least five stars.
 
Zoom back out to the largest field of view, look toward the north again.

\bigskip

{\bf C. Other Constellations.}

Identifying constellations isn't a big part of this course (after
all, there's no physics involved in constellation-spotting).  Still, it's
good to know where a few prominent constellations are.

The two stars at the end of the bowl of the Big Dipper (Merak and Dubhe)
are called the pointer stars.  Draw an imaginary line from Merak to Dubhe,
and extend it about six times the separation between those stars.
You'll hit the star Polaris, which is the tail end of the
Little Dipper (the constellation Ursa Minor).  Polaris is also known
as the North Star.  Polaris isn't a terribly bright star -- in fact,
none of the stars in Ursa Minor are very bright -- but it's still
important.  It's the one star in the sky that all the others seem
to rotate around.  (More on this later.)

Look to the East from Polaris to find the constellation Cassiopeia.  It
looks like a W (on its side at the moment).  The stars in Cassiopeia
are quite bright, and Cassiopeia is easy to spot in the night sky pretty
much all the time.  Cassiopeia and the Big Dipper are probably the most
useful constellations to use when orienting yourself to the night sky.

That's enough with constellations for now; we'll do more when we actually
do some observing.

\bigskip

{\bf D. Daily Motion of Stars.}

Make sure you're looking North, with the largest possible field
of view.  Under ``View Options,'' turn off the box that says ``Daylight.''
This will make it so that the sky is dark during the day, so that
you can see the stars all the time.  (Astronomers would love
it if this were possible in real life!)

Set time to go at 3000x the normal rate and observe how the stars move.
The stars move in circles, with the star Polaris at the center.
Note that some of the stars move in circles that always stay
above the horizon, while others rise and set below the horizon.
Stars that never set below the horizon are called ``circumpolar.''

How many of the five stars in the W of Cassiopeia are circumpolar?

\vskip 1in

Now change locations to Boston, Massachusetts.  How many of the stars
in Cassiopeia are circumpolar when viewed from Boston?

\vskip 1in

When Santa looks at the stars from the north pole, how do they
appear to move?  What percentage of the stars he sees are circumpolar?
Make a prediction first, then try it.

\vskip 1in

Now switch your location to Santiago, Chile.  Find the center of the
circles that stars appear to move in from this location (hint: look South).
What constellation is this in?

\vskip 1in

Note that there is no bright star right at the center of these circles:
Polaris is the North Star, but there is no South Star.

Go back home to Richmond.
Pick a star that's not circumpolar (that is, one that rises and sets).
Record the time that this star rises on one day, and the time it
rises on the next day.  Make sure your times are accurate to the minute.
How much time elapses between successive risings of the star?

\vskip 1in

Repeat for a few other stars.  Is the result always the same?

\vskip 1in


The length of time between successive risings of a star is called a
``sidereal day.''  How different is a sidereal day from  an ordinary
day?

