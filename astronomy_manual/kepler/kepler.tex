\section{Finding Masses with Kepler's Third Law}

\makelabheader

Newton's version of Kepler's third law goes like this:
$$
P^2=\left[ \frac{4\pi^2}{G(m_1+m_2)}\right]a^3
$$
Here $P$ is the period of an orbiting body, $a$ is the semimajor axis
of the orbit, and $m_1$ and $m_2$ are the masses of the central body
and the orbiting body.  $G$ is a constant whose value is
$$
G=6.67\times 10^{-11}\,{\rm m^3/(kg\,s^2)}.
$$
All the stuff after the $10^{-11}$ is just the units of $G$.
It signifies that we're using seconds for our times, kilograms for
our masses, and meters for our distances.

One incredibly useful thing about this form of Kepler's third law
is that it lets you determine the masses of things.  In fact, most
of the time, if we know the mass of an astronomical object, we figured
it out using this law.

The point of this little ``lab'' is to write this law in a convenient
form for determining masses and then use it to figure out 
masses of some astronomical objects.

The first thing to realize is that usually the mass of the central
body ($m_1$) is much, much larger than the mass of the orbiting
body ($m_2$).  That means that, to an excellent approximation, we can
ignore $m_2$ in Kepler's third law and just write it
$$
P^2=\frac{4\pi^2}{Gm_1}a^3.
$$
Since $G$ is a universal constant (always the same known value), 
we can use this to determine the mass of any object, as long as we 
can measure the period and radius of the orbit of something
that's going around that object.

Use this formula to figure out the mass of the Sun in kilograms.  To do this,
consider the orbit of the Earth around the Sun.  What is $P$?
What is $a$?  (Remember that the value of $G$ given above requires
these quantities to be in seconds and meters. You may need to look
up a number or two. That's OK.)

\answerspace{2in}

When using Kepler's third law, astronomers usually don't use kilograms,
meters, and seconds.  Instead, they usually measure periods in years,
lengths in A.U., and masses in units of the Sun's mass (``solar masses'').
In these units, the arithmetic becomes a bit simpler.

Consider the Earth's orbit around the Sun again.  In these new units,
what are $P$, $a$, and $m_1$?  (Hint: you don't need your calculator for
this step!)

\answerspace{1in}

In these units, what is the numerical value of the constant $\displaystyle\frac{4\pi^2}{G}$?
(Plug the values you just wrote down into Kepler's law to see what happens.)

\answerspace{1in}

Remember that $G$ is a universal constant, as are 4 and $\pi$. 
So the combination $\displaystyle\frac{4\pi^2}{G}$ will always have this simple
numerical value, no matter what orbiting system we're considering (as
long as we're using units of A.U., solar masses, and years).

For instance, let's use Kepler's third law to determine the mass
of the Earth.  Remember that the Moon's orbital period is 27.2 days.
What is this in years?

\answerspace{1in}

The radius of the Moon's orbit is 384,000~km.  What is this in A.U.?

\answerspace{1in}

Use these values to determine the Earth's mass in units of solar masses.

\answerspace{2in}

\pagebreak[2]
Earlier, you determined how many kilograms there are in a solar mass.
Use this value to convert your answer for Earth's mass into kilograms.

\answerspace{0.9 in}

The solar system orbits the center of the Milky Way Galaxy in a circular
orbit with a radius of 8.5 kiloparsecs (kpc).  How many meters is this?

\answerspace{0.9in}

How many A.U. is it?

\answerspace{0.9 in}

The solar system's speed in its orbit around the Galaxy is about 220 km/s.
How much time does it take the Sun to make one complete orbit?  (Hint:
how far does it have to travel during one orbit?)  Give your answer in
seconds and in years.

\answerspace{1.9in}

Using your values for the period and radius of the solar system's
orbit, determine the mass of the Milky Way Galaxy in solar masses.

\answerspace{1.9in}

People have done surveys of all the visible material in the Milky
Way Galaxy (stars, gas, and dust).  They estimate that all the
visible stuff has a combined mass of about $1.0\times 10^{10}$ solar
masses.  What percentage of the total mass of the Galaxy is visible
stuff?  What percentage is stuff we don't see?  (The latter
stuff is called ``dark matter.'')


\answerspace{1in}
