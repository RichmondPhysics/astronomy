\chapter{Spectra of Light Sources}

\bigskip\bigskip

In this lab 
you will use a small portable spectroscope to examine several sources
of light.  The spectroscope consists of a slit at one end that lets
light through and a ``grating'' at the other end that splits the light
up into the various colors.  We'll worry a bit later about exactly
how the grating works.

To set up the spectroscope, hold it so that the slit is vertical.  Point
it at a bright light source and look through the other end.  Rotate
the grating end (keeping the slit vertical) until you see horizontal
bands of color on either side of the slit.  These are the colors that
make up the light coming through the slit.

Examine the following light sources with the spectroscope:

\begin{enumerate}

\item The white light coming from the overhead projector.
\item The light from a candle flame.
\item The light from one of the two gas discharge lamps.
\item The light from the other gas discharge lamps.
\item The light from the fluorescent lights in the hallway.
\item The light from the Bunsen burner flame, while someone is placing
a small amount of the white crystalline stuff in it.

\end{enumerate}

Sketch the spectra of all the light sources, then
answer the following questions:.


\begin{enumerate}
\item Which sources have ``continuous'' spectra, meaning that the
light consists of all colors from red to blue?

\vskip 1in

\item For the sources with continuous spectra, can you notice
a difference in the relative amounts of the various colors?
For instance, does one have more red light relative to yellow
or blue, as compared with the other?  (This may be hard to discern,
because the total intensity of the sources is different.)

\vskip 1in

\item Using a chart of spectra that I'll give you, try to determine what gases
are in the two discharge tubes.  Also try to determine
what's in the white powder.

\vskip 1in

\end{enumerate}

