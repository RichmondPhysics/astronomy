\chapter{Galaxy Rotation Curves and Dark Matter}

\paragraph{Introduction.}

As you know by now, the way we (almost always) measure masses
in astrophysics is by measuring orbits and applying Kepler's Third
Law.  When we apply this technique to galaxies, we get a total
mass that is much larger than the mass of all of the visible
stuff we see.  This is the main way astrophysicists infer that
galaxies contain large amounts of unseen ``dark matter.''

In this lab, you will use observations of the spiral
galaxy NGC 3521 to figure out the mass distribution of the galaxy.
The data you will use was taken by Y. Sofue and collaborators
with a radio telescope in Japan.\footnote{In case you're
wondering, further information about the data is at
http://www.ioa.s.u-tokyo.ac.jp/$\sim$sofue/h-rot.htm,  but you don't
need to do anything with the information at that site.}
(This is real data, unlike
the various simulation labs we often do.)  

\paragraph{Useful facts for this lab.}

You'll need to use the small-angle formula,
$$
s={\alpha d\over 206,265}.
$$
You'll also need Kepler's Third Law,
$$
P^2={4\pi^2a^3\over G(m_1+m_2)}.
$$
Remember that in units of years, AU, and solar masses, the
constants $4\pi^2\over G$ come out equal to one and so can
be ignored.  Finally, some unit conversions:
$$
1\,{\rm pc}=206,265\,{\rm AU}.
$$
$$
1\,{\rm AU}=1.49\times 10^8\,{\rm km}.
$$


\paragraph{Procedure.}

\begin{enumerate}

\item Download and open the Excel file {\bf NGC 3521 rotation curve},
which is in the {\bf Downloads for labs} section of the Blackboard
site for this course.  This file contains measurements of the
radial velocity (the speed towards or away from us) of gas clouds
in this galaxy as a function of the clouds' distance from the center
of the galaxy.  The first column is the apparent location of the cloud
relative to the galaxy's center.  The fourth column is the speed
of that gas cloud towards or away from us.  (The speed is towards
us on one side of the galaxy and away on the other side.  Both
sides have been combined in this data set.)

\item How do you think that the speed information was obtained?

\vskip 1in

\item You'll need to know the actual distances of the various gas clouds
from the center of the galaxy.  Use the small-angle formula to figure
those out in both kiloparsecs and AU, entering the results in the
second and third columns of the spreadsheet.  The distance to the
galaxy NGC 3521 is 8900 kpc.  Since there are many rows of data, you'd
be very unwise to do this step one row at a time; use an Excel formula
to do them all at once.  See Appendix C for reminders about how to do this sort
of thing.

\item Plot a graph showing the distance from the center in kpc on the
$x$ axis and the rotation speed on the $y$ axis.  Be sure that
the axes of your graph are clearly labeled including both the
name of the quantity being plotted and its units.  This graph
is called the ``rotation curve'' of the galaxy.

\item Using another Excel formula, 
create a new column in the spreadsheet giving the period
of the orbit of each gas cloud in years.  (The orbits can
all be assumed to be circles.)  As usual, be sure the column
is labeled with both a name and units.

Your formula will probably give an error in the very first line,
because the radius and speed are both zero there.  That's OK,
as long as it gives correct results for the other lines.  The
same thing is true for the next step.

\item Next you will use Kepler's Third Law to determine the mass contained
within each gas cloud's orbit.  Create a new column to contain
this information, and use an appropriate Excel formula.

\item Plot a graph showing the distance from the galaxy's center
in kiloparsecs on the $x$ axis and the mass on the $y$ axis.

\item From your graph, it should be clear that something has gone 
wrong at large distances.  Explain how you know.  (What does the graph
do at large distances?  Why is this impossible?)

\vskip 1.5in

As it turns out, the data are reliable out to about 17 kpc.

\item Other measurements have been made to determine the 
distribution of light coming from this galaxy.  They have determined
that 90\% of the total light from the galaxy comes from the
innermost 9 kpc of the data.  Assuming that the maximum value
of mass you found is the total mass of the galaxy, what percentage of the
total mass of the galaxy lies within 9 kpc of the center? 
Based on this information, would you conclude that the total mass
in the galaxy is more spread out than the stars, or less spread out?

\newpage
%\line{\ }

\vskip 1.5in

\item In answering the previous question, you assumed that the
largest mass value you found was equal to the total mass of the galaxy.
That might be wrong: the graph of mass might continue to go up if
we had reliable data at greater distances.  If you did discover
that the total mass of the galaxy continued to go up as you observed
out to greater distances, would it strengthen or weaken
the conclusion you drew in the previous question?

\vskip 1in

\item The luminosity of this galaxy is about $1.5\times 10^{10}$
times the luminosity of the Sun.  Calculate the ``mass-to-light
ratio'' of this galaxy by dividing the total mass (in solar masses)
by the total luminosity (in solar luminosities).  

\vskip 0.5in

\item If the galaxy were made
up of nothing but stars just like the Sun (with no gas, dust,
or dark matter), what would the mass-to-light ratio be?

\vskip 0.5in

\item Astrophysicists think that a typical collection of gas,
dust, and stars in a galaxy (with no dark matter) 
should have a mass-to-light ratio of about 2.  Assuming that
that's correct, use the luminosity to 
determine the mass of all of the gas, dust, and
stars in this galaxy.  How many times greater is the total mass
of the galaxy than the mass of all the visible stuff?

\end{enumerate}


