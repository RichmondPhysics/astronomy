\setcounter{chapter}{3}
\setcounter{page}{14}
\chapter{Daily Motion of Stars}
% (Stellarium version)}

This lab has a few purposes:
\begin{itemize}
\item To get used to some features of the {\it Stellarium}
program.
\item To identify some prominent constellations and asterisms
that can be seen from our location.
\item To see how stars move in the night sky.
\end{itemize}

\bigskip

{\bf A. Messing around.}

Start up {\it Stellarium}, and spend a few minutes playing around with
some of its features. Appendix \ref{app:stell} lists a bunch of things
you can do. Here are some specific things you should try. Refer
to the Appendix for details about how do do these things.

\begin{enumerate}
\item Adjust the time and date. When you first start the 
program, it will show the sky at the current time.  It's much
more interesting to see the sky at night. You can
also examine what things look like at different times of year.

\item Speed up the flow of time by a large amount, so that
a day goes by in just a few seconds.

\item Initially, the program is showing you the view of the sky
from here in Richmond. Switch to some other locations.

\item Click on a star or other astronomical object.
Information about that body should appear on the screen.
Some of that information may not make sense yet, but it will!

\item Drag the image around (hold down the left mouse button
and move the mouse around) to look at different parts 
of the sky.

\item Zoom in to look at a small patch of the sky, then
zoom back out
(using the mouse wheel or the page-up/page-down keys).
Note the display at the bottom that says ``FOV.'' This stands
for ``field of view,'' and it indicates the size of the patch of sky
that's visible on the screen at that moment. Watch how the FOV changes
as you zoom in and out.

\item Click on the ``find'' icon and search for various celestial
objects. 
Try Uranus, for example, or Polaris (which
is the name of the North Star).


\item The pop-up menus in the lower left
contain a bunch of buttons you can click to change
the appearance in various ways. I've listed the ones I think
are most useful in the Appendix. Try these out. Some are harder
to understand than others.

\end{enumerate}

\bigskip

{\bf B. Big Dipper, Mizar, and Alcor.}

Once you're done messing around, set your location back to Richmond,
looking north, with a nice, large field of view (say $90^\circ$ or so).
Set the
time to 10:00 tonight.

You should see the Big Dipper in the northeast.
Turn
on the constellation labels and lines.  Notice that the Big
Dipper is not a full constellation; it's just part of the larger constellation
Ursa Major (the Great Bear).  A group of stars like the Big
Dipper, which is easily
identifiable and has a name, but which isn't a whole constellation,
is called an ``asterism.''  

The second star in the handle of the Big Dipper is called Mizar.
Click on this star to select it, and move it into the center
of the field of view. 
Then zoom in on this star until you can see
another star called Alcor right near it.  Find the angular separation
between these two stars.
What is the angular separation?

\vskip 1in

This separation is visible to the naked eye, if you have good eyesight.
Next time you're out on a clear night, look to see if you can spot both
Alcor and Mizar.

Zoom in on Mizar still further, and you'll eventually see that it
splits into two stars. Amazingly enough, although \textit{Stellarium}
might not tell you this, each of those two stars is itself a double
star system.
Unfortunately, these two pairs are too close together for us to 
see them separately, even with our best telescopes.  (How do we know
that they're there, then?  Good question! We'll answer it eventually.) 
In fact, in 2009 it was discovered that Alcor is actually a double
star system as well.
So when you look at Mizar you're really seeing four stars,
and if you look at both Alcor and Mizar, you're seeing six stars.

What is the angular separation between Mizar and its companion (not
Alcor -- the closer one)?

\vskip 1in

Select Mizar by clicking on it (if it's not still selected). 
You'll see some information about it in the upper left corner,
including the distance to the star in light-years. 
Use the small-angle formula to determine the separation between
Mizar and its companion.  The small-angle formula will give you
an answer in light-years.  Convert this to meters and to astronomical
units.


\vskip 2in

 
Zoom back out to the largest field of view, look toward the north again.

\bigskip

{\bf C. Other Constellations.}

Identifying constellations isn't a big part of this course (after
all, there's no science involved in constellation-spotting).  Still, it's
good to know where a few prominent constellations are.

The two stars at the end of the bowl of the Big Dipper (Merak and Dubhe)
are called the pointer stars.  Draw an imaginary line from Merak to Dubhe,
and extend it about six times the separation between those stars.
You'll hit the star Polaris, which is the tail end of the
Little Dipper (the constellation Ursa Minor).  Polaris is also known
as the North Star.  Polaris isn't a terribly bright star -- in fact,
none of the stars in Ursa Minor are very bright -- but it's still
important.  It's the one star in the sky that all the others seem
to rotate around.  (More on this later.)

Look to the West from Polaris to find the constellation Cassiopeia.  It
looks like a W (on its side at the moment).  The stars in Cassiopeia
are quite bright, and Cassiopeia is easy to spot in the night sky pretty
much all the time.  Cassiopeia and the Big Dipper are probably the most
useful constellations to use when orienting yourself to the night sky.

In the winter, the other easy constellation to spot is Orion. Find
it in the southern sky.

That's enough with constellations for now; we'll do more when we actually
do some observing.

\bigskip

{\bf D. Daily Motion of Stars.}

Make sure you're looking North, with a large field
of view (at least $90^\circ$).  Turn off the effects of Earth's 
atmosphere.
This will make it so that the sky is dark during the day, so that
you can see the stars all the time.  (Astronomers would love
it if this were possible in real life!)

Set time to go at much faster than the normal rate,
so that a day takes only a a few seconds.
Observe how the stars move.
The stars move in circles, with the star Polaris at the center.
Note that some of the stars move in circles that always stay
above the horizon, while others rise and set below the horizon.
Stars that never set below the horizon are called ``circumpolar.''

Of course, whether a star sets below the horizon or not depends on
the exact shape of the horizon. To keep things simple, let's assume that
we're looking at the stars from a location with a nice, flat horizon
(no hills or trees to get in the way). Here's one way to make this happen
in \textit{Stellarium}: click on ``Sky and viewing options'', then
the ``Landscape'' tab, then check the ``Ocean'' box. As you'll
see, that shows you what things would look like if you were surrounded
by a nice, flat ocean.

How many of the seven main stars in the
Big Dipper are circumpolar? (Here I'm counting all of
Alcor and Mizar as one star.)


\vskip 1in

Now change locations to Boston, Massachusetts.  How many of the stars
in the Big Dipper are circumpolar when viewed from Boston?

\vskip 1in

When Santa looks at the stars from the north pole, how do they
appear to move?  What percentage of the stars he sees are circumpolar?
Make a prediction first, then try it.

\vskip 1in

Now switch your location to Santiago, Chile.  Find the center of the
circles that stars appear to move in from this location (hint: look South).
What constellation is this in? (There are several constellations right near
this point. Astronomers have defined precise boundaries between the
constellations. To see which one this point is in, figure out how to turn
on the ``constellation boundaries'' in \textit{Stellarium}.)

\vskip 1in

Note that there is no bright star right at the center of these circles:
Polaris is the North Star, but there is no South Star.

Go back home to Richmond.
Pick a star that's not circumpolar (that is, one that rises and sets).
Record the time that this star rises on one day, and the time it
rises on the next day.  Make sure your times are accurate to the minute.
How much time elapses between successive risings of the star?

\vskip 1in

Repeat for a few other stars.  Is the result always the same?

\vskip 1in


The length of time between successive risings of a star is called a
``sidereal day.''  How different is a sidereal day from  an ordinary
day?

