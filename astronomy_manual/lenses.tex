\chapter{Image formation by lenses}


In this lab, you will examine how lenses form images.  In particular,
you will test the relationship between the location of the object
whose image is being formed and the location of the image.  Theoretically,
we expect that relationship to be
$$
{1\over p}+{1\over q}={1\over f}.
$$
Here $p$ is the {\it object distance} (distance from object to lens),
$q$ is the {\it image distance} (distance from image to lens, and $f$
is the {\it focal length} of the lens.

\bigskip

{\bf Part 1: Focal Length of a Lens}.

\begin{enumerate}

\item Arrange the light source appartus to produce parallel
rays of light across the surface of a piece of paper.
\item Place a convex lens in front of the
light rays.  Mark the lens position on a piece of paper, and trace
the path of the rays.
\item What is the distance from the center of the lens to the focal
point?  This is the focal length of the lens.

\vskip 1in

\end{enumerate}

\bigskip

{\bf Part 2: Image Formation}

\begin{enumerate}

\item Attach the light source to one end of the optical bench,
and place the lens partway down the bench.  Arrange the light
source so that the side with a circle and arrow drawn on it is
facing the lens.  This will be the ``object'' whose image we will
be examining with the lens.

\item Measure the length of the arrow.  We will call this $h_o$, meaning
``the height of the object.''

\vskip 1in

\item Adjust the lens and an index card until you see a clear, in-focus
image of the object on the card.  Measure the height of the image
$h_i$, the distance from lens to object $p$, and the distance from
lens to image $q$.  Record them in the table below.

\item Move the lens about 5 cm toward or away from the object, and move the
card until you get a clear image again.  Measure $h_i,p,q$ again.
Repeat until you have five sets of measurements.  For some, the image should
be closer to the lens than the object ($q<p$) and some should be 
the other way around ($p<q$).  (Note: make sure that the object
is always further away from the lens than the focal length.)

\vspace{0.3cm}
{\centering \begin{tabular}{|c|c|c|c|c|c|}
\hline 
~~~~~~~\( p \)~~~~~~~&
~~~~~~~\( q \)~~~~~~~&
~~~~~~~\( h_{i} \)~~~~~~~&
~~~~~~~\( \frac{h_{i}}{h_{0}} \)~~~~~~~&
~~~~~~~\( \frac{q}{p} \)~~~~~~~&
~~~~~~~\( f \)~~~~~~~\\
\hline
\hline 
&
&
&
&
&
\\
\hline 
&
&
&
&
&
\\
\hline 
&
&
&
&
&
\\
\hline 
&
&
&
&
&
\\
\hline 
&
&
&
&
&
\\
\hline
\end{tabular}\par}
\vspace{0.3cm}

\item For each observation, calculate and record the ratio of
image and object heights, $h_i/h_o$, and the ratio of image and
object distances, $q/p$.  Record these
in the table.  What can you conclude about these
quantities?

\vskip 1in

\item For each observation, use the formula at the beginning
of this lab to calculate the focal length of the lens, and record
it in the last column of the table.  Do these results appear consistent
with your measurement of the focal length in part 1?

\vskip 1in

\item Move the lens closer to the object, so that the object distance
is less than the focal length.  Note that there is nowhere you can
place the card to get a clear image.  Look through the lens at the
object.  Does the image appear to be larger or smaller than the actual
size of the object?  Does it appear to be closer or further away?

\vskip 1in

\item Suppose the object is extremely far away from the lens.
Based on your results, where would you expect the image to form?
(In other words, about what would you expect $q$ to be?)  Would
you expect the image to be large or small?

\vskip 1in

\item If the Sun is shining, take the lens outside along with
a piece of white paper, and try to form an image of the Sun on the
paper.  Are the results consistent with your predictions?

\vskip 1in

\end{enumerate}


