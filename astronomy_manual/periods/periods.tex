\section{Sidereal and Synodic Periods}

\makelabheader

\medskip

\paragraph{Part 1. An example.}
Imagine two runners, Ellen and Peter,
running around a circular track, starting out
next to each other. Each runner runs at a constant speed, but one 
is faster than the other: Ellen takes 8 minutes to go around the track
once, and Peter takes 10 minutes. Let's call these two times
$E$ and $P$ respectively: $E=8$ minutes and $P=10$ minutes.


Eventually, Ellen will ``lap'' Peter -- that is, she will have gone around
the track one time more than him, so that she's right
next to him again. I want
you to figure out how much time it takes for this to happen.
There are different ways to do this, but one thing that might help
is to figure out where Ellen and Peter are after 10 minutes have gone by.
By how much (what fraction of a lap) will Ellen will be ahead of 
Peter? How many times would you have to repeat this until Ellen
had lapped Peter?

Your final answer should be a number of minutes. This is the amount
of time it takes for the two runners to go from a particular configuration
(right next to each other) to the same configuration again. If Peter and Ellen
were planets, this would be called the ``synodic period'' of the two planets.

\answerspace{2in}


We're going to want a general relationship giving the synodic period
in terms of the ``sidereal periods'' $P$ and $E$. It turns
out that the relationship is simplest if we express it in terms
of the reciprocals of these numbers.

What are the reciprocals of the three numbers $P$, $E$, $S$ for
this situation? (Express them as decimals, not fractions.)
$$
\frac{1}{P}=
\hspace{1.7in}
\frac{1}{E}=
\hspace{1.7in}
\frac{1}{S}=
\hspace{ 1.7in}
$$

Can you spot a relatively simple mathematical relationship
between these three numbers? Write it down in the form of an equation
involving $\frac{1}{P}$, $\frac{1}{E}$, $\frac{1}{S}$.


\answerspace{2in}


\paragraph{Part 2. The general rule.}
The rule you wrote down in the previous part turns out to apply
no matter what $P$ and $E$ are. We're going to check that this
is true algebraically now. In this section, you should forget 
the specific numerical values for $P,E,S$ (that is, the 10 minutes, 8 minutes,
etc.). We're going to work out an expression that is true
for different possible values of these quantities.

As before, $E$ stands for the time that Ellen takes to go around
the track once, and $P$ stands for the time that Peter takes. $S$
stands for the ``synodic period'' -- that is, the time it takes for
Ellen to pull ahead of Peter by one full lap.

After a time $S$ has gone by, how many laps has Ellen run? Your
answer will be an algebraic expression involving $E$ and $S$.\footnote{If
you don't know what to do, try thinking it through with a particular
number in mind. For instance, suppose that it takes Ellen 2 minutes
to run a lap (so $E=2$ minutes), and suppose $S=20$ minutes. How
many laps will Ellen have run in those 20 minutes? 
Write down whatever you did to get your answer, but use the letters
$S$ and $E$ in place of 20 and 2, and you have the expression I want.}

\answerspace{1.5in}

After a time $S$ has gone by, how many laps has Peter run?
Your answer will be an expression involving $S$ and $P$. 

\answerspace{1.5in}

\pagebreak[4]

By definition, $S$ is the time at which Ellen has run one more
lap than Peter. That means that your answers to the two previous
questions must differ by one. Write down that last sentence in
the form of an algebraic equation.

\answerspace{1in}

Do some algebraic manipulations on that equation until
it looks like the equation you wrote down at the end of Part 1.



\vfil

You're done! You've just derived the general rule relating sidereal
and synodic periods. When we apply this in astronomy, $E$ will
be the sidereal period of the Earth (one year), $P$ will be the sidereal
period of another planet (the length of a year on that planet), and 
$S$ will be the synodic period.

A couple of quick notes:
\begin{itemize}[nosep]
\item In this derivation, we assumed that Ellen was faster than Peter.
That means that the formula we derived only works if the other
planet has a longer period than the Earth. That turns out to be true
for all of the planets further out than the Sun (Mars, Jupiter, etc.).
The planets Mercury and Venus, go faster than the Earth, so the
final formula looks slightly different. You could work it out by the
same sort of reasoning if you wanted to, or you can look it up in
your textbook.
\item In this expression, we usually give all of the periods in years, but
you don't have to. You can use any unit of time you want, but it has
to be the same unit for all three quantities.
\item The main reason this relationship is useful is in determining
sidereal periods for other planets. After all, we know $E$ (one year),
and $S$ is easy to measure -- just observe the time it takes between
two successive times the planet goes into opposition. So the
most common use of this rule is to find the one remaining quantity,
namely $P$.
\end{itemize}




