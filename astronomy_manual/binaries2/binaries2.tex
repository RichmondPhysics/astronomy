\section{Spectroscopic Binary Stars}

\makelabheader


\paragraph{Part 1: Determining the masses of spectroscopic
binary stars.}

%In your last lab, you worked through one method of determining
%the masses of the stars in a binary system. That method involved
%observations of the stars' spectra (to get the Doppler shifts) and
%also images of the stars (to determine how far apart they were).
For many binary star systems, we can't actually resolve the two stars (i.e,
see
them as separate objects); all we can do is observe the combined light
from both of them. 
%In that case, the method you used last time
%doesn't work. Today, 
In this lab, you'll work through a calculation to figure
out the masses of a pair of \textit{spectroscopic binary} stars,
in which we can observe the spectra of the pair of stars together
but can't see each of the stars individually.

Suppose that you measure the spectra of such a binary star system, and you
find that it has a pair of spectral lines, both very close to the wavelength
656.28 nm, which is the wavelength of a prominent spectral line in hydrogen.
At any given moment, one is redshifted to a longer wavelength,
and the other two is blueshifted to a shorter wavelength. As time passes,
each line cycles back and forth between redshift and blueshift.

You make some measurements of the spectral lines and observe the 
following facts:

\begin{enumerate}
\item At a moment when the lines have the largest Doppler shifts,
one of them has a wavelength of 656.72 nm, and the other
has a wavelength of 655.40 nm. 
\item Half a cycle later (i.e., 5 hours later),
the wavelengths are 655.84 nm and 657.16 nm.
\item Another 5 hours later, the spectral lines are back at the wavelengths
they were at at the first moment. In other words, the time
it takes for the lines to cycle all the way around and back to where
they started is 10 hours.
\end{enumerate}

Sketch a picture showing the orbits of the two stars. Indicate
where the stars are at Moment 1 and at Moment 2, and indicate the location
of you the observer.

%\vskip 2in
\answerspace{2in}
%\vfil\eject


What are the speeds of the two stars? (You can use the wavelengths
at either of the two moments described above. You might want to use
both just to check your work.)

\answerspace{2in}

\pagebreak[2]

You know the speeds of the two stars, and you know how long it takes
them to go around once. From this information, determine the radii
of the two stars' orbits. You can assume that the orbits are circles.
(If you're not sure what to do, I'll give the one-word hint
``circumference''. If you still don't know, of course, ask!)

\answerspace{2in}

The quantity called $a$ in Kepler's Third Law is the distance between
the two stars. How is this related to the two radii? (You may
want to refer to your picture at the beginning of this lab.) Now that you know
that relationship, tell me the value of $a$ for this system.

\answerspace{2in}

Use Kepler's third law to determine the total mass $M_1+M_2$ of the
two stars. Give your answer in solar masses. I strongly recommend
that you do this by determining $a$ and $P$ in units of AU and years
respectively. Recall that Kepler's third law takes a relatively simple
form in this case.

\answerspace{2in}

\pagebreak[2]

Now determine the individual masses of the stars. To do this, make
use of the fact that the speeds at which the stars orbit
are related to their masses like this: $M_1v_1=M_2v_2$. Combine
this fact with the value you found for the total mass, and solve for
$M_1,M_2$.

\answerspace{2in}

In fact, what you've just discovered is the \textit{minimum} possible
mass for the two stars: they may in fact be heavier than this.
Can you think of a reason that your calculations may have given
masses that are smaller than the true values? (Can you think of an
assumption you made in your calculations that may not be true?)

\answerspace{1in}

\paragraph{Part 2: Finding the masses of planets orbiting other stars.}

One of the hottest topics in astrophysics these days is the discovery
of planets orbiting other stars. One of the ways people do this is
by looking for the ``wobble'' of the star caused by the planet
orbiting it. The planet itself is too small and dim to be observed,
so we must infer its presence from the star's motion.

For example, observations of the star HD209458 reveal the following:
\begin{enumerate}
\item The
velocity of the star wobbles back and forth in a regular way, repeating
every 3.525 days. 
\item The star's maximum speed towards us is 87.1 m/s,
and its maximum speed away from us is 87.1 m/s in the other direction.
(Note that these are meters per second, not kilometers per second.)
\item The star is a main-sequence star very similar to the Sun.
Its mass is $1.13M_\odot$.
\end{enumerate}

You can use this information to figure out the mass of the planet
orbiting this star, even though you can't see the planet or measure
its spectral lines.

First, use Kepler's third law to determine the radius of the
planet's orbit. I recommend that you use the AU-year-solar-mass
version of the law. You can assume at this point that the
mass of the planet is so small that it can be ignored in comparison
with the star's mass -- that is, $M_1+M_2$ can be taken to be the
same as the mass of the star.

%\vfil\eject

Now that you know the radius of the planet's orbit, you can figure out
the speed of the planet in its orbit. (How far does it travel during
one orbit? How much time does this take?)

\answerspace{2in}

Now remember the rule $M_1v_1=M_2v_2$ from above: the masses of the two
bodies in an orbit are related to their orbital speeds. You know both
speeds, so you can express the mass of the planet in terms of the
star's mass. What is the mass of the planet in solar masses?

\answerspace{2in}

Planet-hunters often express the masses of the planets they find
in units of Jupiter's mass. Jupiter's mass is about $\frac{1}{1000}$
that of the Sun. What is the mass of the planet in units of Jupiter masses?

\answerspace{1in}

By the way, just as in part 1, the mass you've found is actually
the minimum possible mass, for the same reason. But subsequent
observations show that, in fact, the problem that might result
in this calculation giving an underestimate of the true mass
isn't really a problem in this case: the mass you've found is about right.
If you want to know how we know, I'll tell you.


