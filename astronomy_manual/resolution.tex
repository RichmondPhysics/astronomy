\chapter{Resolution of the Human Eye}

Everyone knows that there are limits to how well we can see. Some
objects are too small or too close together for our eyes to distinguish
them. The same goes for any telescope or other observing device we
might use: even the Hubble Space Telescope has limits to the level of fine
detail it can ``see.''
We define the \textit{resolution} of an observing device (an eye
or a  telescope,
for instance) to be the \textit{angular size} of
the smallest details it can see. The smaller the resolution, the
better the device is at seeing fine details.
In this lab, you will measure the resolution of your own eyes.

You will be given a piece of paper that has a pair of black dots on it,
along with a single oval-shaped blob that's about the same size as the
pair of dots. If you look at the paper from a short distance, it's
easy to tell the pair of dots apart from the single blob, but from
a great distance you can't tell which is which.
We say that you can \textit{resolve} the pair of dots if you
can tell that the two dots really are two dots, not one.

\paragraph{Measurements.}
Take the sheet with the dots on it and move quite far away (10 meters
or so) from your lab partner. Hold the sheet up either right-side up
or upside-down, but don't tell your partner which. From this
distance, your partner should not be able to tell the difference
between the pair of dots and the single blob.
Your partner
should then walk slowly toward you until he or she can tell which 
one is pair of dots. Measure the distance between the two of you
at the moment your partner can first tell with confidence
which is which.

Repeat this procedure at least three times. Switch the orientation
of the paper randomly, so that your partner doesn't know the ``right''
answer. Switch roles, so that each partner is the ``observer'' 
at least three times.
List the individual measurements below, and
also compute the average distance for each partner.

\vfil\eject
%\line{\ }

%\vskip 2.5in

The resolution of your eyes is the angular separation between the
two dots when you can just barely tell that there are two of them.
To figure out this number, you need to know the 
separation between the two dots (to be precise, the separation between
their centers). Measure this separation.

\vskip 1in

Use the above information to calculate the resolution of each
partner's eyes. Give your answer in degrees and also in
arc-minutes. Remember that one arc-minute ($1'$) is $1\over 60$ of
a degree.

\vskip 2.5in

Once you've got your answers, record them on the whiteboard,
so that everyone can see the range of resolutions.

Now suppose that you repeated the experiment using dots that
were twice as big, and twice as far apart.
What would you expect the distance between partners to come out to be?

\vskip 1in

What would you expect the resolution to come out to be?

\vskip 1in

The back of the page contains dots that are twice as large.
Try the experiment and see. (You don't need to do 3 repetitions per partner
this time; just one is fine.) Are the results consistent with
your predictions?

\vskip 1in

\paragraph{Questions.}

Here are some questions to test whether you know what you're doing
with all this stuff. Some of these questions have to
do with resolution, and some are other applications of
the small-angle formula.
You don't have to turn these in, but 
you just might have a quiz some time soon that asks questions very similar
to these.

Note that I've obnoxiously used a variety of different units here.
If you need to look up unit conversions, go ahead. 
In some cases, I may have deliberately left out a number you need, in which case
you should estimate it as best you can.

(In case you're wondering, on a quiz or exam, I will not expect you to know 
numbers such as the number of meters
in a light-year or the distance from Earth to Sun.
You should know the basic
metric-system prefixes, like the number of centimeters in a meter
or meters in a kilometer.)

\begin{enumerate}

\item The Moon is 384\,000 kilometers away. There are two large
craters on its surface, which you can just barely distinguish from
each other with the naked eye. How far apart are the craters?

\item At the Battle of Bunker Hill, the rebels were supposedly
told, ``Don't shoot until you see the whites of their eyes.''
How close would the British be at this point?

\item You're standing on a deserted road, late at night. A vehicle
is approaching you. At first, you can't tell whether it is a car (with
two headlights) or a motorcycle (with one). How close will the vehicle
be before you can tell? 

\item The resolution of the Hubble Space Telescope is about $0.1''$ (that's
0.1 arc-seconds). How many times better than your eye is this?

\item The two dwarf planets Pluto and Charon (formerly known as the planet
Pluto and its moon) are about 20\,000 kilometers
apart. They are about 29 astronomical units from the Earth. What is
the resolution of a telescope that can just barely resolve these two
bodies? Give your answer in arc-seconds.

\item Suppose that another star somewhere nearby has a planet
orbiting it at exactly the same distance as the Earth is from the Sun.
How close would the star have to be in order for the Hubble
Space Telescope to be able to see the planet and star as separate
objects? Give your answer in light-years.\footnote{Actually,
seeing other planets is even harder than this question suggests.
Looking for something very faint right next to something
very bright is extremely difficult.}
The closest star is 4.2 light-years
away. Is this close enough?

\item One of the stars in the handle of the Big Dipper is actually
a pair of stars called Alcor and Mizar. The two stars have
an angular separation of about $12'$ and are about 1 light-year
away from us. How far are the two stars from each other?
Give your answer in meters and in astronomical units.

\item The supergiant star Betelgeuse has an angular diameter of $0.044''$
 and is 427 light-years from Earth. What is Betelgeuse's
diameter? If the Sun suddenly swelled up until it was as big as Betelgeuse,
which of the planets of the solar system would it engulf?


\item A typical distance between neighboring galaxies is 2 million
light-years. A good ground-based telescope has a resolution of about $2''$.
How far away must a galaxy be if an observer using this telescope
has difficulty resolving it (that is, telling it apart from its
neighbors)? Because the Universe is not infinitely old, there's
a maximum distance we can see of about 50 billion light-years (light
from greater distances hasn't had time to reach us). Bearing this
in mind, is there ever a problem telling galaxies apart from their
neighbors with a telescope like this?

\item 
\begin{quote}
``Riders!'' cried Aragorn, springing to his feet. 
``Many riders on swift steeds are coming towards us!''

``Yes,'' said Legolas, ``there are one hundred and five. 
Yellow is their hair, and bright are their spears. Their leader is very tall.''

Aragorn smiled. ``Keen are the eyes of the Elves,'' he said.

``Nay! The riders are little more than five leagues distant,'' said Legolas.

{\hfill J.R.R. Tolkien, \textit{The Two Towers}}
\end{quote}

Roughly what is the resolution of Legolas's eyes? 


\end{enumerate}

%\newpage

\paragraph{Answers.} 

For purposes of the answers below, I'll assume that the resolution
you found for your eye was $2'$ (two arc-minutes). 

\begin{enumerate}

\item 220 km.
\item I'll estimate that the white of your eye is about 1 centimeter
across (from the edge of your iris to the edge of your eye). Something
about that size can be resolved at a distance of about 20 meters.
\item $2'$ is $120''$, which is 1200 times bigger than $0.1''$, so 
the Hubble space telescope has a resolution about 1200 times better than
your eye.
\item Assuming the headlights are about 2 meters apart, the distance
comes out to about 3.4 kilometers.
\item About $0.95''$.
\item 33 light-years. So the nearest star is close enough.
\item $3.3\times 10^{13}$ m, or 220 AU.
\item The diameter is $8.6\times 10^{11}$ m or 5.7 AU. All
planets within half this distance, or about 2.85 AU, 
of the Sun would be engulfed. That's Mercury, Venus, Earth, and Mars. 
(Jupiter's orbit is about 5 AU in radius.)
\item About $2\times 10^{11}$ light-years. This is about 200 billion
light-years, which is much bigger than the maximum distance
we can see, so telling galaxies apart from their neighbors
isn't generally a problem even for very distant galaxies.
\item We've got to make some estimates here. Let's say that
Legolas can resolve features that are as small as 0.1 meters. (If
his resolution were much worse than this, then I don't think he could
tell a tall person from a short person. Naturally, if you used
a somewhat different number, that's fine.) 
Five leagues is about 15 miles, or about 25 km.
The resolution corresponding to these numbers is $0.0002^\circ$, or $0.85''$.
Legolas does about as well as a good astronomical telescope.

\end{enumerate}
