\chapter{Energy Flow in the Sun}


In this lab, you will examine how the energy produced in the core
of the Sun makes it out to the surface.  The nuclear reactions in
the core of the Sun produce photons, which eventually make their way
to the surface of the Sun.  This process takes a long time because
the photons undergo frequent collisions with ionized atoms
in the solar interior.  These collisions cause the photons to
bounce around in random directions instead of zooming
straight out.

You will use simulation software to examine the process by which
these photons make it out of the Sun.
If it's not already running, 
start up the ``energy flow out of the Sun'' application on the lab
PC and log
in.  It doesn't matter what you enter for names and table number.\footnote{For 
boring technical reasons, this program has to run inside of a 
``virtual machine'' that's set up on the lab computers. I'll make sure
that it's all set up before class. This may mean that the application
may run in a window inside of another window, which looks a little funny
but doesn't make a difference in the end. }

\medskip

{\it Part 1: Photons moving through a gas.}

Select ``Interaction'' under the ``Simulation'' menu, and click
``Run.''  This shows what happens to photons as they pass through a
dense gas.  The stationary circles represent atoms in the gas, and the
smaller white circles represent photons.  You should imagine that
there is a source of photons off to the left somewhere, sending
photons into the gas.  These photons have the correct amount of energy
to correspond to one of the spectral lines of the atoms.  That means
that when a photon hits an atom, it is very likely to be absorbed,
sending the atom into an excited state.  The atom quickly drops
back down to its original state, re-emitting the photon in a new direction.
The end result is that the photon undergoes a ``random walk'': instead
of moving in a straight line, it bounces around making frequent
changes in direction.

Stop the simulation.  Under ``Photon Type,'' select ``Continuum,'' and
restart the simulation.  With this setting, the photons now have an
energy that does {\it not} correspond to a spectral line of these
atoms.  This means that the atoms cannot absorb the photons, so the
photons usually just pass right through.  

On rare occasions, a photon
with the wrong energy can bounce off of an atom, as you'll see if you 
let the simulation run for a while.

The {\it mean free path} is the average distance that a photon travels
before it bounces off of an atom.  Which has a longer mean free path,
the continuum photons or the line photons?

\vskip 1in

If the continuum photons {\it never} bounced off of atoms (as opposed
to rarely doing so), what would their mean free path be?

\vskip 1in

Here's an important thing to realize for the rest of the lab:
Throughout most of the interior of the Sun, the gas is ionized,
meaning that the electrons are stripped off of the protons.  Ionized
gas is capable of interacting with photons of {\it all} wavelengths,
not just specific ones.  That means that {\it in the interior of the
Sun, all photons behave like the ``line'' photons in this simulation.}

{\it Part 2: Photon Diffusion in the Sun}.

Click ``Return'' to get back to the main menu.  Under ``Simulation,'' 
go to ``Flow'' and then to ``1 Photon.''  
This simulation shows the path of a single photon as it makes its
way out from the center of the Sun.  Click ``Run'' a few times to 
see how the photons behave.  Go to ``Parameters'' and check the ``Yes''
box under ``Trails.''  This will cause the photon's path to be
visible as it makes its way out.

In the ``Parameters'' menu, you can also vary the size of the Sun.
Select \# of Layers.  It should initially be 20.  Change that value to 40
and try running the simulation again.  You should see that the Sun
has gotten larger, and not surprisingly it takes more time for the photons
to make their way out.

In this model of the Sun, each ``layer'' of the Sun has a thickness
equal to the mean free path of the photons.  That is, on average,
a photon can make it in or out of the Sun by only one layer between
bounces.  The more layers there are, the more times a photon is likely
to bounce on its way out, and so the longer it will take to get out.
In fact, there is a general rule relating the number of layers to the
time it takes photons to get out, which you will now determine.

Because the photons bounce around randomly, some get out faster than others.
We're going to be interested in the time it takes the \textit{average}
photon to get out, so we'll want to run the simulation with a bunch of photons
all at once. Go to the ``Simulation'' menu, and click ``Flow'' and then
``Diffusion.'' This will behave just like the simulation, but with many photons.


Under ``Parameters,'' you can set the number of layers and the number of
photons. Set the number of layers to 10, and the number of photons to 200.
Start the simulation, and let it keep going until all 200 photons have escaped
from the Sun.
record
the average number of interactions required for a photon to make it out.

\vskip 1in


Repeat the process for a 20-layer Sun.  


\vskip 1in



Do it one more time, this time for a 30-layer Sun. This one will take
a little while!

\vskip 1in

Based on these results, take a guess about the mathematical relationship
between the number of layers and the average number of interactions
taken to get out:

\vskip 1in

Ask me if you're not sure whether your guess is right.

This relationship only tells you about the {\it average} time it takes
a photon to get out.  As you've seen,
some photons are faster than average, and some
are slower. Let's do one more simulation to get a feel for this.

Set the number of layers to 25, and the number of photons to 1000,
and start the simulation running.  After how many interactions
does the first photon make it out?

\vskip 1in

Wait until most of the photons have escaped.  What is the average number
of interactions of all the photons?  Is this consistent with your
expectations based on the mathematical relationship you guessed above?

\vskip 1in

The graph in the lower part of the window shows the number of photons
escaping the Sun as a function of time.  The yellow line is the time
when the number of interactions is $n^2$ (the square of the
number of layers).  Sketch a copy of the graph below.

\vskip 2in

This graph shows that quite a few photons (well over half) make it out
in less than the average time, but some take much, much longer than that.

\medskip

{\it Part 3: Numbers for the actual Sun.}

In these simulations, the number of layers is very small.  In the actual
Sun, the photons can travel only a very short distance before
scattering (that is, the ``mean free path'' is small).  Since a ``layer''
is supposed to be only as thick as the mean
free path, a realistic model of the Sun would have many, many layers.
In this portion of the lab, you'll work out some rough numbers for
the actual Sun.

Based on the physical laws governing interactions between photons
and electrons, scientists estimate that the mean free path
of photons inside the Sun is about a tenth of a millimeter ($10^{-4}$ meters).
The Sun's radius is about 700,000 kilometers.  If
each ``layer'' has a size equal to the mean free path, how many layers
does the Sun consist of?

\vskip 1in

Based on the relationship you found earlier between the number of layers
and the number of interactions (bounces), how many interactions
will an
average photon undergo on its way out of the Sun?

\vskip 1in

The average distance a photon travels between interactions
is equal to the mean free path, and of course photons travel at the
speed of light (300,000 km/s).  What is the average time between
interactions for a photon in the Sun?

\vskip 1in

From the average time between interactions and the total number of
interactions, determine how much time it takes the average photon to
make it out of the Sun.  Convert your answer into years.

\vskip 1.5in

If the nuclear fusion reactions powering the Sun were to somehow stop
tomorrow, would we notice an immediate reduction in the Sun's
luminosity?

\vskip 1in

Is there any way we could tell if the nuclear reactions in the Sun
stopped tomorrow?


