\section{Black Holes}

\makelabheader

\paragraph{Mass and radius.}
One of the key facts about black holes is the mathematical relationship
between the radius of the black hole and its mass:
$$
R_{\rm Sch}=\frac{2GM}{c^2}.
$$
The radius is called the ``Schwarzschild radius'' after Karl Scharzschild,
who first worked out the mathematics of black holes.  First, let's work out
Schwarzschild radii for some different sized black holes.  It's convenient
to group all of the constants together like this:
\begin{equation}
R_{\rm Sch}=\left(\frac{2G}{c^2}\right)M.
\end{equation}
Remember that the gravitation constant is $G=6.67\times 10^{-11}\,{\rm
m^3/(kg\,s^2)}$ and the speed of light is $c=3.00\times 10^8$ m/s.
What is the numerical value of the combination $(2G/c^2)$, and what
are its units?  

(Remember how to determine the units: substitute the values of $G$
and $c$ into the expression $2G/c^2$, with their units, and then manipulate
the units algebraically just as if they were numbers or variables or anything
else -- for instance, cancel anything that appears in both numerator and
denominator of a fraction.)

\answerspace{0.3in}
$$
\frac{2G}{c^2}=\qquad\qquad\qquad\qquad
$$
\answerspace{0.3in}

According to equation (1) above, you can convert any given mass into
the corresponding Schwarzschild radius by simply multiplying by this
value.  Are the units you found above consistent with this statement?
What units do you need to use for the mass?  What about for the radius?

\answerspace{1in}

What is the Schwarzschild radius for a black hole the mass of the Earth 
($5.98\times 10^{24}$ kg)?

\answerspace{1in}

A typical black hole that formed from a star has a mass of about 5 times
the mass of the Sun.  What is the Schwarzschild radius for such
a black hole?
(The Sun's mass is $1.99\times 10^{30}$ kg.)

\answerspace{1in}

\pagebreak[2]

The black hole at the center of our Galaxy may have a mass as high as 
$4\times 10^6$ solar masses.  What is its Schwarzschild radius?

\answerspace{1in}

Suppose we wanted to make a black hole whose radius was $10^{-10}$ meters
(about the size of an atom).  How much mass would we need?

\answerspace{1in}

\paragraph{Density.}
One way to get a feel for the extreme conditions required to form
a black hole is to consider the density of matter involved.
If we wanted to produce a black hole of a certain mass, we'd have
to compress all of that matter into a sphere whose radius was
equal to the Schwarzschild radius.  Remember that density is
mass divided by volume, and that the volume of a sphere is
$$
V=\frac{4}{3}\pi R^3.
$$

Using this information, 
write down an expression giving the density required to form a black
hole in terms of the mass $M$.  Your expression should just have $M$
and constants such as $G$ and $c$ in it.  In
particular, it should not involve the radius $R$ --- 
if it does, use equation (1) to 
express the radius in terms of $M$.

\answerspace{2in}

Density = 

\answerspace{0.5in}

Note: It's not really correct to say that this is the {\it density
of the black hole} (although sometimes people do say this).  Rather,
it's the density of matter required to cause the black hole to form in
the first place.  After the black hole forms, all the mass gets concentrated
into a very small volume at the center, so the final central density after the
black hole has formed is much higher.

Does a very massive black hole require a larger or smaller density
than a not-so-massive black hole?

\answerspace{1in}

What is the density of matter required to form a 5-solar-mass black hole?
(Be sure to include units!)

\answerspace{1in}

What is the density of matter required to form a million-solar-mass
black hole?

\answerspace{1in}

What would the mass have to be if the required density is equal to the
density of water?  (Note: the density of water is 1 gram per cubic
centimeter, but grams per cubic centimeter are not the units you 
want to use in this expression!  If you're not sure how to express
the density in the correct units, ask me.)

\answerspace{2in}


You meet an alien from the Andromeda galaxy, who asks you the following
question: ``To make a black hole with a mass of 10 slurms, the required
density is 50 quatloos.  What density is required to make a black
hole with a mass of 40 slurms?''  You don't know how large a unit of 
mass a slurm is, nor do you know how large a unit of density a quatloo is.
Nonetheless, you can answer the alien's question.  What is the answer?

\answerspace{2in}



\paragraph{Orbital Periods.}
Black holes often have smaller objects orbiting around them.  When they do,
we can use the orbital periods to determine the black hole's mass
using Kepler's third law.  Remember Kepler's third law:
$$
P^2=\frac{4\pi^2a^3}{G(m_1+m_2)}
$$
For the next couple of calculations, it may help you to recall that
1 AU is $1.50\times 10^{11}$ meters and that 
1 year is $3.155\times 10^7$ seconds.

Suppose you observe a possible supermassive black hole at the center
of a galaxy.  A star is orbiting the center with an orbital radius of 1 AU
and is going at a speed of $6\times 10^7$ m/s.

What is the star's orbital period?

\answerspace{2in}

What is the mass of the black hole?

\answerspace{2in}

Express the radius of the star's orbit as a multiple of the 
black hole's Schwarzschild radius.  


