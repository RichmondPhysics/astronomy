\chapter{Apparent Motion of the Planets}

The motion of the stars in the night sky is pretty simple: they
go in circles about the north celestial pole, making one complete
circuit per sidereal day. The motion of the Sun and Moon are a bit
more complicated: they share in the daily motion of the stars, but
they gradually drift from west to east relative to them. In other words,
as the stars race around the north celestial pole, the Sun
and the Moon gradually fall behind.

The motion of the planets has a lot in common with the Sun and Moon,
but it's a bit more complicated. One of the most important scientific
advances in all of human history was Copernicus's correct
explanation of why the planets seem to move the way they do. 
To understand what Copernicus figured out, we first have
to examine how the planets appear to move in the sky.

\paragraph{Jupiter's daily motion.}
The first thing to realize is that, on any particular day, the
planets seem to move in pretty much the same way as the stars: they
circulate about the north celestial pole, rising in the east and setting
in the west, taking about one day to go all the way around. Let's start
by making sure of this.

%% Start up \textit{Sky Safari}.\footnote{If you prefer, you 
%% can do everything in this lab using \textit{Stellarium.} The main thing
%% to remember there is that the ``equatorial and horizon coordinate systems''
%% in \textit{Sky Safari} correspond to the ``equatorial and azimuthal
%% mount'' in \textit{Stellarium}.}
Start up \textit{Stellarium}.
Locate the planet Jupiter. If it's not up above the horizon during the night
time, shift time forward a month at a time until it is. Once you've
reached a time when Jupiter is up at night,
let
time run rapidly for several days, 
and observe Jupiter's motion. You
should find that, if you don't look too closely anyway, it moves in
pretty much the same way as the stars it's next to.

Determine, to an accuracy of about a minute or so, the time when Jupiter sets
on two successive days. How much time elapsed between these two occurrences?

\vskip 1.5in

If you did the same thing with a star (i.e., measured the difference between
two times  the star set), what would you find? You can try it if you want,
but I hope you know what the answer is.

\vskip 1in

The two answers should be very similar. Depending on how carefully you 
measured, you may or may not have found a small difference between them.
Now let's examine that small difference more carefully.

\paragraph{Jupiter's slow drift with respect to the stars.}
Over the course of a few days, Jupiter (and the other planets) seem to
move in approximately the same ways as the stars, but there are 
small differences, which build up to become quite important
over longer periods of time. To examine those differences,
it helps to ``turn off'' the daily motion of everything.
To do this, switch to the ``equatorial mount.''
You should also turn off the effects of daylight,
so that you can watch the planets at all times.


Set the time to the present, locate Jupiter, and center it in the field
of view. Set time running forward \textit{very} rapidly, so that a year goes by
every few seconds. You should see Jupiter generally drift from right
to left with respect to the stars, but sometimes reverse itself
and drift from left to right.\footnote{If you're keeping Jupiter 
centered in the field of view, of course, it won't actually move at all.
When I say you'll see Jupiter go from right to left
with respect to the stars, what I really mean
is that you'll see the stars slip past it from left to right.}
When we look at the sky in the equatorial (stationary-sky) point of view, 
east is always on
our left and west is on our right, so we say that Jupiter
usually goes from west to east, but sometimes reverses direction and
goes from east to west.

The times when Jupiter goes ``backwards'' (east to west) are called
periods of \textit{retrograde motion}. The next thing we want to 
do is look for patterns in when Jupiter goes into retrograde motion.

Set the time back to the present, and start it running forward
rapidly. Find the beginning and end of the next three time periods
when Jupiter is in retrograde motion. It's hard to spot the
exact moment when retrograde begins or ends. You don't
have to get the exact date right -- just determine the month and
year. To get you started, I'll tell you that Jupiter will
next go into retrograde motion in about April 2019.


While you're at it, find all the times 
when Jupiter will be
in \textit{conjunction} with the Sun during this period (from
now through the end of the third retrograde period). 
This means the time when
it passes right next to the Sun in the sky.

List your results (three time periods when Jupiter is in retrograde
and all the moments when it's in conjunction) below.

\vskip 3in

On a blank sheet of paper, make a time line covering the entire
range of all these times, and the periods of retrograde and conjunction.

Summarize your results in a sentence or two.

\vskip 1in

\paragraph{Motion of other planets.}
Try the same thing with the planets Mars and Venus.
To be specific, identify the next three times when the planet is 
in retrograde, and all the times of conjunction from now until the end of those
three
retrograde periods. Make a time line for each.

\vskip 5in

Which of these two planets shows a pattern very similar to Jupiter?
Which is different?

\vskip 1in

Make a guess about why one of these planets is not like the others.

\vskip 1in



