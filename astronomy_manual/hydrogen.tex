\chapter{The Spectrum of Hydrogen}


In this lab, you will use a spectrometer to measure the wavelengths of 
the spectral lines of hydrogen.  The spectrometer consists of a slit
through which light passes, a diffraction
grating, and a small telescope to focus the light that comes out
of the grating.  The telescope can be rotated in a circle centered
on the diffraction grating.  By measuring the angles at which the
spectral lines appear in the telescope, you can determine the
wavelengths of the lines.

The relationship between the angle and the wavelength is
$$
d\sin\theta = n\lambda.
$$
Here $d$ is the spacing between lines on the diffraction grating, 
and $\theta$ is the angle through which the light from a spectral
line is bent by the grating.
In this spectrometer, we are always seeing the ``first-order'' diffraction
pattern, so $n=1$.
If you don't know or don't remember your trigonometry, don't worry
about it: all you need to know for this lab is how to make your
calculator tell you the sine of an angle.

(Incidentally, this formula is essentially the same as the formula
you used in an earlier diffraction lab, although that one was
written in non-trigonometric form.)

The first thing you'll need to know is $d$, the spacing between
lines on the grating.  The grating should be labeled with a number of
lines per inch or a number of lines per millimeter.  Use this 
to determine the distance between lines in either
meters or centimeters.
(If your
grating is labeled in lines per inch, it may help you to know that one
inch is 2.54 cm.)

\vskip 2in

Now that you know $d$, you can determine the wavelength of any spectral
line just by measuring the angle $\theta$ through which the light is bent.

Set up the spectrometer with the hydrogen lamp right against the slit.
Put the grating at the center, so that it is oriented perpendicular to
the direction from the slit.  Look through the telescope, and
gradually move the telescope around until you see the spectral lines
from the lamp.  You don't want the bright spot that you see when the
telescope is lined up pointing straight at the lamp; the spectral
lines you're interested in are visible when it's off to the side.

Hydrogen has four spectral lines in the visible part
of the spectrum.  You should be able to see three of them clearly.
The fourth one is fainter, but you may be able to see it.

For each of the three brightest spectral lines, determine
the wavelength as follows.  

\begin{enumerate}
\item Adjust the spectrometer so that each spectral
line is centered in the telescope, and measure the angle $\theta$
corresponding to each spectral line on both the left and the right.
Note that the angles marked on the spectrometer have $180^\circ$
for light that is not deflected at all.  We'd rather
call that angle $0^\circ$ instead of 180$^\circ$, so you should
subtract $180^\circ$ from your angles.

Measure the angles as accurately as possible.  Note that there is 
a ``Vernier scale'' allowing you to measure an extra decimal place.

\item For each of the three lines, average the two measurements of
$\theta$ together, and use the resulting value to determine $\lambda$.

\item For each of the spectral lines, record the energy of a photon,
using the rule 
$$
E={hc\over\lambda}.
$$
Planck's constant has the value $h=4.135\times 10^{-15}$ eV s, and
the speed of light is $c=3.00\times 10^8\,{\rm m/s}$.  If you use these
values, and if the wavelength is measured in meters, your energy
will come out in units of electron volts (eV).
\end{enumerate}


\begin{center}
\begin{tabular}{|c|c|c|c|c|c|}\hline
Line        & $\theta_{left}$     & $\theta_{right}$     & $\theta_{average}$ & Wavelength  & Photon Energy \\ 
\hline
& & & & & \\
Line 1 (reddest)  &                             &                              &                    &                    &   \\ \hline
& & & & & \\
Line 2   &                             &                              &                    &                    &   \\ \hline
& & & & & \\
Line 3 (bluest)  &                             &                              &                    &                    &   \\ \hline
\end{tabular}
\end{center}

It turns out that there is a pattern in the energy levels of hydrogen.
The energy levels are of the form
\begin{eqnarray*}
E_1&=&-{A\over 1^2},\\
E_2&=&-{A\over 2^2},\\
E_3&=&-{A\over 3^2},\\
&\ldots&
\end{eqnarray*}
Here $A$ is a constant, whose value you are to determine.

The longest-wavelength visible spectral line of hydrogen
(the red one) occurs when an atom jumps from energy level $n=3$
down to $n=2$.  So the energy of one of those red photons should be
$$
E_3-E_2=-{A\over 3^2}-\left(-{A\over 2^2}\right)=-{A\over 9}+{A\over 4}.
$$
Set this expression equal to the energy you determined for the red spectral
line, and solve for the value of the constant $A$.  Show me the
result when you're done.

\vskip 2.5in

Once you know the value of $A$, use it to determine the energy of a photon
emitted when a hydrogen atom jumps from level 4 to level 2:

\vskip 1in

From level 5 to level 2:

\vskip 1in

These should look similar to the energies you meaured for the other
two spectral lines.  If they don't, talk to me.

What is the energy of a photon corresponding to the next spectral line
(a jump from level 6 to level 2)?  What is the wavelength of 
this spectral line?

\vskip 1.5in

Using the formula $d\sin\theta=\lambda$, predict the value of $\theta$
at which this spectral line should appear in the spectrometer.
[Note to the trig-averse: to do this, you'll need to use the ``inverse
sine'' (or ``arcsine'' or ``sin$^{-1}$'') function on your calculator.]
Now that you know where to look, see if you can see this spectral line
in the spectrometer.

\vskip 1in

One final question: Why do all of these spectral lines correspond
to jumps down to the second energy level?  Why not the first or third,
for instance?


