\chapter{Celestial Navigation}

In the old days, before GPS, sailors figured out where they were
by looking at the positions of the stars. In this lab, you'll
see how that worked. To be specific, you'll look at the question
of how to use the positions of celestial 
objects to determine your latitude and longitude.

Our reason for doing this is not that I expect you to find yourself
lost at sea without access to GPS (although anything's possible,
I guess). It's because
figuring out how celestial navigation works is a good way to make
sure you understand how things move in the night sky.

\section*{Latitude}

As it turns out, determining your latitude (i.e., how far north or
south of the equator you are) is much easier than determining your longitude.
The easiest way is to observe the location of the star Polaris.
As you know, Polaris is very close to the North Celestial Pole, or in
other words almost directly above the Earth's North Pole. For
purposes of the questions below, you can assume that Polaris is
exactly at the North Celestial Pole.

Suppose you were standing at the Earth's North Pole, which is at a latitude
of $90^\circ$ north. In which direction would you have to look
in order to see Polaris?

\vskip 1in

Now suppose you measured the \textit{altitude} of Polaris. 
As we've seen, the 
altitude of an object is the angle that describes how far above
the horizon that object is located. To be specific, draw an imaginary
line from you to Polaris, and draw an imaginary horizontal line (pointing
toward the horizon) directly below Polaris. The altitude means
the angle between those two lines.

If you were at the north pole, what would the altitude of Polaris be?

\vskip 1in

Now suppose you were at the equator (which is at a latitude
of $0^\circ$). In which direction would you have to look in order to see
Polaris?

\vskip 1in

What would the altitude of Polaris be?

\vskip 1in

Based on the above considerations, you can guess that there might
be a \textit{very} simple rule relating the altitue of Polaris
to the observer's latitude. What do you think that relationship is?

\vskip 1in

We could test this observation using actual observations of Polaris at
night, but instead let's test it with \textit{Sky Safari}. 

Start up \textit{Sky Safari}. 
Find the star Polaris and select it. Look at the information in
the upper left, and find the star's altitude. What is the altitude of
Polaris?

\vskip 1in

What is our latitude here in Richmond? (You can find this out by looking
in the 
``Location'' area under ``Settings'' in \textit{Sky
Safari}, or I'm sure Google knows it.)
\vskip 1in

Do the two values you just found agree, at least roughly, with your
expectations? (If the degrees agree but there's a discrepancy in the
arc-minutes, that's close enough.)

\vskip 1in

If you're ever lost at sea, you now know how to find your latitude,
by observing how high Polaris is in the sky.

\section*{Longitude}

Suppose you wake up one night and find yourself in a boat tossing about
in the middle of the ocean. Using the method above, you figure out
your latitude. Now how are you going to find your longitude? As we'll see,
this
turns out to be a much harder problem than finding latitude.

Suppose your
latitude comes out to be $20^\circ$ north. 
To keep things (relatively) simple, let's suppose that there
are just two possibilities
for longitude: either you're at a longitude of $25^\circ$ west,
or you're at a longitude of $55^\circ$ west. 

Set the time in \textit{Sky Safari} to about 10:00 tonight,
and set your location to be a latitude of $20^\circ$ north and
a longitude of $25^\circ$ west. (You can type these numbers
directly into the Location box. Just enter the
number of degrees, and leave out the minutes and
seconds.) Orient your view so that you're looking north,
with the horizon at the bottom and a pretty large 
field of view (about $60^\circ$ or so). Note a couple of
landmarks to orient yourself in the sky. You should be able
to see the Big Dipper (upside-down) and the bright star Capella, for
instance.

Once you've got this set, take a screen shot by holding down
the master power switch on the side and hitting the home button
on the front.
Check that the screen shot was saved (under ``Photos'').
Open it up and take a look at it to make sure it looks
the way you expect. 
Once you see that the file looks right, go back to \textit{Sky Safari}.

Now change your location to $55^\circ$ west longitude (keeping
the latitude and the time the same).
Take a screen shot of the sky from this new location.

Now advance the time by two hours, from 10:00 pm to midnight.
Take one more screen shot.

At this point, you should have three screen shots.
\begin{enumerate}
\item Longitude: $25^\circ$. Time: 10:00 pm.
\item Longitude: $55^\circ$. Time: 10:00 pm.
\item Longitude: $55^\circ$. Time: midnight.
\end{enumerate}
Incidentally, I should mention that the times in \textit{Sky Safari}
are always times in our actual location (Richmond). That is, 10:00 pm
means 10:00 pm Eastern time, regardless of the observer's location.

Two of these three images should look almost identical, and one
should look significantly different. Which one is not like the others?

\vskip 1in

\newpage

Now, let's get back to your plight as you sit bobbing in your boat
in the middle of the ocean. Suppose that you have a timekeeping
device (wristwatch cell phone, etc.) that is set to Eastern
time. Using this, along with your observation of the night
sky, can you tell which longitude you're at? If so, explain
briefly how. If not, explain briefly why not.

\vskip 3in

Suppose that you \textit{don't} have an accurate timekeeping device
on board the boat. Can you tell what longitude you're at by
observing the sky? If so, explain
briefly how. If not, explain briefly why not.

\vskip 3in

In the 18th century, the British government offered large
cash prizes for anyone who could figure out an accurate way
to determine the longitude of a ship at sea. 
In 1765, John Harrison was awarded a \pounds 10\,000 prize
for solving this problem. (It's hard to
figure out precise equivalents, but this is 
equivalent to  millions of dollars today.) 
What do you think Harrison invented?

\vskip 1in

Back to you on your boat. Suppose that you had a clock with you
on your boat, but it wasn't very accurate. Suppose that you use the clock
to determine your longitude, but unbeknownst to you the clock is off by
two hours. (For instance, you think it's midnight, when really it's 10:00
pm.) By how many degrees will your longitude determination be off?

\vskip 1in

Suppose that your clock were only off by one minute instead. How far
off will your longitude determination be?

\vskip 1in

One degree of longitude corresponds to an actual distance of about 50 
kilometers. If your clock were off by one minute of time, 
how far off would your determination of your location be (in kilometers)?

\vskip 1in

If you were a ship's captain 
trying to avoid hitting undersea rocks
and shoals, this level of inaccuracy would be a real problem.

