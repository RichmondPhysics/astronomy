\section{Introduction to Sky Safari}
\newcommand{\skysaf}{\textit{Sky Safari$+$}}
%\newcommand{\icon}[1]{\includegraphics[height=0.2in]{figs/stell-#1.eps}}

\bigskip\bigskip

\skysaf\ is an iPad / Android app that simulates the appearance
of astronomical objects in the sky. The user can adjust lots of
features of the view, including the observer's location,
the time, etc. We'll make heavy use of it as we try to understand
how objects appear to move in the night sky.

Although it's certainly not required, 
I encourage you to download it onto any suitable device
you own. \skysaf\ is not free, but it doesn't cost much.
There is a free version
of \textit{Sky Safari} that does many but not all of the things we'll want.

In this Appendix, I'll list some of the features of the program that will
be most useful to us. The best way to learn about these features is not
just to read about them here, though. Spend some time playing around, to learn
what you can do.


\paragraph{Moving around the sky.} 
This all works in the way that you're surely used to.
Drag across the screen to change what area you're looking at,
and zoom in and out using the usual two-fingered pinching motion.

Note that the ``Field of View'' (FOV) is indicated
at the upper right (something like $99.3^\circ\times 76.1^\circ$, for
instance). The FOV is the size (in degrees) of the
patch of sky that's visible on the screen at that moment.

\paragraph{Selecting and centering on an object.}
You can select a star or planet (or other celestial object)
by tapping it.
If you hit the ``Info'' button at the bottom,
you'll get a bunch of information about the object
(more than you wanted to know).
If you hit the
``Center'' button, the object will be moved
to the center of the screen. Once an object has been centered, it will
stay centered until you choose a new object or drag to view
a different part of the sky.

\paragraph{Changing the observer's location.} Hit
``Settings'' and then select ``Location'' to 
choose where the observer is located. You can either enter latitude
and longitude or choose a city from a list. 

\paragraph{Changing time.} Hit the ``Time''
button to bring up a set of options for adjusting the time.
The ``Now'' button, unsurprisingly, will return the time to the present.
The button to the right of ``Now'' will advance the time forward
by one step. The buttons in the next row let you choose the step size to be a year, month, day,
hour, minute, or second. The button to the left of ``Now'' steps
backwards.
The triangular buttons on the left and right make time advance 
steadily, with a speed that depends on the step size you've chosen.
In case it's not clear what this means, select ``Hour'' and then
hit the triangular button. Then select ``Minute'' or ``Day'' and
see what happens.

The date and time are indicated in the upper right.

\paragraph{Finding a specific object.} Hit the search button
and enter the name of the object you're looking for.

\paragraph{Miscellaneous display options.} 

You can find all of these under the ``Settings'' menu.

The ``Horizon \& Sky'' tab under ``Display Options'' lets you customize
how the Earth's surface is displayed. If you turn off the ``Show 
Horizon \& Sky'' option, then you'll get to see how things would
look if the Earth were invisible. That is, you'll get to see even
the stars that lie below the Earth's surface. 
You can also turn on and off the effects of daylight. With daylight turned
off, you can see the stars as they would appear during the day.
Of course, these are things we can't do in real life
(no matter how much astronomers wish we could), but they're
useful tools for helping to build intution about how
objects move in the sky.

By default, the program is set up to show you the horizon as a 
(somewhat) realistic landscape. Sometimes it's nicer to see what
things would look like if we were surrounded by a completely flat horizon
(no hills, trees, etc.). To do this select the option to show
the horizon as an ``opaque area.''

The ``Constellations'' tab under ``Display Options'' lets you
change how constellations are indicated.
In particular, you can turn on or off the display of names
of constellations (``Show constellations''). You can also have
the program display lines connecting the stars in a constellation or even
indications of the mythical figures that the constellations represent.
Personally, I don't think the last one is very helpful, but
some people like it.

Astronomers label points in the sky with two kinds of coordinates, 
which \skysaf\ calls ``horizon coordinates'' and ``equatorial coordinates.''
In other sources, you'll often see ``horizon coordinates'' called
``azimuthal coordinates.''
We'll talk about these in detail in class,
but for the moment here are the basics.
Both of these are like latitude and longitude on Earth. The
difference is that the equatorial grid is ``attached'' to the sky, so
that each star's coordinates stay the same as time passes.
The
horizon (or azimuthal)
grid, on the other hand, is attached to the Earth. 
The ``Grid and Reference'' tab under ``Display Options''
lets you turn on and off a grid showing either one of these coordinates.


\paragraph{Measuring angles.}
Often, we want to know the \textit{angular separation} between two
objects in the sky. Angular separation is the angle between a line from your eye
to one object and a line from your eye to the other object; it's a measure
of how far apart the objects appear to be. To determine
angular separation, select the the first object, then
select the second, and then hit the ``info'' button. Near
the bottom of the information screen, you'll find the angular separation.

For instance, to find the angular separation between Jupiter and Saturn,
you'd tap Jupiter, then Saturn, then ``info.'' Then you'd
look for the line that says
``Angular separtion from Jupiter.'' (It'd say ``from Jupiter''
because it's currently displaying information \textit{about}
Saturn. In general, you want the line that refers to 
angular separation from the first object you tapped.)

\paragraph{Stationary-Earth and stationary-sky points of view.}
As we'll see in detail, objects in the night sky appear to rotate
approximately once per day about a point in the sky called the ``north
celestial pole.'' This apparent motion is due to the 
Earth's rotation on its axis: the stars are really (more or less)
sitting still. 
Often, it's useful to examine what the sky would look like if
we could see things from a point of view in which 
the Earth's rotation has been taken away, so that the sky appears
to sit still. 

In \skysaf, this is called switching to the ``equatorial coordinate system.''
To do this, go to ``Settings,'' and then find the ``Coordinates''
tab (under ``Time \& Coordinates''). If you hit ``Equatorial,''
then the display will be shown in the point of view in which the sky sits
still. If you hit ``Horizon'' (which is probably where you started),
then you'll see things in the usual point of view in which the Earth
sits still.

For reference, it may help to know that other sources call
these two points of view ``azimuthal mount'' (stationary-Earth)
and ``equatorial mount'' (stationary-sky). The reason for the word
``mount'' is that these two points of view describe the two
ways that telescopes are generally set up (``mounted''). A
telescope with an equatorial mount is set up with a motor that turns
the telescope at just the right rate to cancel out the Earth's
rotation, so that it shows us the stationary-sky point of view.

\paragraph{Finding things in the actual sky.}
You can use \skysaf\ to locate objects in the night sky. Naturally,
this only works outdoors at night, so we won't do this in class, but if you
install it yourself you may want to play around with it. Tap the
``compass'' button and hold the device up. It'll show you
an image of the sky in the
same direction as you're looking in real life. This is very helpful
in identifying objects in the sky. If you want to find a given
object in the sky, search for it in \skysaf\ and hit ``center'' while
the compass is on. \skysaf\ will indicate with an arrow which direction 
you should turn to find the object. Keep on following the arrows
and eventually you'll be facing (and holding your device) in a way
that points right at the object you're looking for.
